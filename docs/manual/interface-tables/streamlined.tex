\newcommand{\tabsecheader}[1]{\multicolumn{2}{c}{\bf #1\hspace{5.5em}\mbox{ }}}
%
\begin{tabular}{lp{0.55\textwidth}}
%\hline
%\bf METHOD  & \bf DESCRIPTION \\
%%--------------------------------------------------------
 \toprule
\tabsecheader{Configuration (optional)} \\\midrule
%--------------------------------------------------------
\begin{lstlisting}
hoppetSetExactDGLAP(threshold,split)
\end{lstlisting}
&\fn
Sets use of exact NNLO mass thresholds and splitting functions
  (default: both false).
  \\\hline
\begin{lstlisting}
hoppetSetApproximateDGLAPN3LO(variant)
hoppetSetSplittingN3LO(variant)
hoppetSetN3LOnfthresholds(variant)     
\end{lstlisting}
&\fn
  Sets variants for the choice of \ntlo splitting functions and
  thresholds, cf.\ Sec.~\ref{sec:n3lo-evolution}.
\\\hline
% 
\begin{lstlisting}
hoppetSetQED(withqed,qcdqed,plq)
\end{lstlisting}
&  \fn Sets QED evolution and its options
  (cf.~Sec.~\ref{sec:qed-evolution}; default: all false).
\\
%%--------------------------------------------------------
  %\hline\hline
  \toprule
\tabsecheader{Initialisation} \\\midrule
%--------------------------------------------------------
%\midrule
% \multicolumn{2}{c}{\bf Initialisation}  \\
%\tabsecheader{Initialisation}\\
%\multicolumn{2}{|c|}{\bf Initialisation\hspace{5.5em}\mbox{ }}  \\
%\midrule
\begin{lstlisting}
hoppetStart(dy,nloop)
\end{lstlisting} 
& \fn
Sets up a compound grid with
spacing in $\ln 1/x$ of \ttt{dy} at small $x$,
extending to $y = 12$ and numerical
order $\ttt=-6$. The $Q$ range for the tabulation will be $1\GeV <
Q<28 \TeV$, \fn~ \ttt{dlnlnQ=dy/4} and the factorisation scheme is
  ${\overline{\rm MS}}$ (\ttt{factscheme\_MSbar}).
  \\
\midrule
\begin{lstlisting}
hoppetStartExtended(ymax,dy,Qmin,
  Qmax,dlnlnQ,nloop,order,factscheme)
\end{lstlisting} & \fn
  More general initialisation. \\
\midrule 
\begin{lstlisting}
hoppetSetFFN(fixed_nf)
hoppetSetPoleMassVFN(mc,mb,mt)
hoppetSetMSbarMassVFN(mc,mb,mt)
\end{lstlisting} &
\fn Set heavy flavour scheme ($\MSbar$ available only to NNLO).\\
%            &\\[-0.5em]
%--------------------------------------------------------
\toprule
\tabsecheader{Normal evolution} \\\midrule
%--------------------------------------------------------
\begin{lstlisting}
hoppetEvolve(asQ0,Q0alphas,
           nloop,muR_Q,LHAsub,Q0pdf)
\end{lstlisting} &
\fn PDF evolution: specifies the coupling \ttt{asQ0} at a 
scale \ttt{Q0alphas}, 
% \texttt{\footnotesize nloop,muR\_Q,LHAsub,Q0pdf)}&
  the number of loops for  evol., \ttt{nloop},
  the ratio (\ttt{muR\_Q}) of ren. to fact. scales,
 the name of a subroutine \ttt{LHAsub(x,Q,f}) that fills \ttt{f(-6:6)},
 and the scale
\ttt{Q0pdf} at which one starts the PDF evolution.
 Note:  \ttt{LHAsub} is only called at scale \ttt{Q0pdf}
and in \CPP \ttt{f[iflv]} spans \ttt{iflv=0..12}.
\\
%The PDF at a given value of  is obtained \\
%(multiplied by $x$) at the given $\ttt{x}$ and $\ttt{Q}$ values \\
%\midrule&\\[-0.5em]
%--------------------------------------------------------
\toprule
\tabsecheader{Cached evolution} \\\midrule
%--------------------------------------------------------
%\bf Cached evolution & \\
\begin{lstlisting}
hoppetPreEvolve(asQ,Q0alphas, 
                nloop,muR_Q,Q0pdf)
\end{lstlisting} & \fn
 Preparation of a cached evolution.\\
\midrule
\begin{lstlisting}
hoppetCachedEvolve(LHAsub)
\end{lstlisting} &
\fn  Perform cached evolution with the initial condition
     at \ttt{Q0pdf} from a routine \ttt{LHAsub} 
with LHAPDF-like interface.
 Note: \ttt{LHAsub} only called at scale \ttt{Q0pdf}.\\
%--------------------------------------------------------
\toprule
\tabsecheader{Evaluation} \\\midrule
%--------------------------------------------------------
\begin{lstlisting}
hoppetEval(x,Q,f)
\end{lstlisting} &
\fn On return, \ttt{f(-6:6)} contains all flavours of the PDF set
                   (multiplied by $x$). In \CPP, the array
                   indices span 0 to 12. Increase upper bound by 5
                   with QED.\\
\midrule
\begin{lstlisting}
hoppetEvalSplit(x,Q,iloop,nf,pf)
\end{lstlisting} &
\fn On return, \ttt{pf(-6:6)} contains the (cached) convolution of the
                   \ttt{iloop} splitting function ($1=\text{LO}$) with the tabulated
                   PDF for the given \texttt{nf}. One can chain splitting
                   functions up to $\order{\as^4}$, e.g.\ \texttt{iloop=31} gives
                   $P_\text{NNLO}\otimes P_\text{LO} \otimes f$.\\
\midrule
\begin{lstlisting}
hoppetAlphaS(Q)
\end{lstlisting} &
\fn Returns the coupling at scale $Q$. \\  \midrule
%
\begin{lstlisting}
hoppetWriteLHAPDFGrid(basename,
                          pdf_index)
\end{lstlisting}
            & \fn
              Write an LHAPDF grid file to \ttt{basename\_nnnn.dat} where
              \ttt{nnnn} is \ttt{pdf\_index}; if
              \ttt{pdf\_index} is
              0, also write a template \ttt{basename.info} file.\\
%--------------------------------------------------------
\toprule
\tabsecheader{Cleanup (optional)} \\\midrule
%--------------------------------------------------------
\begin{lstlisting}
hoppetDeleteAll()
\end{lstlisting}
            & \fn
              Deletes all storage allocated by the streamlined interface
%--------------------------------------------------------
  \\\midrule
\end{tabular}
