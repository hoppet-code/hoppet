\documentclass[preprint,1p,a4paper,11pt]{elsarticle}

\usepackage{amsmath}
\usepackage{booktabs,cellspace}
\usepackage{amssymb,amssymb,mathtools}
\usepackage{url}
\usepackage{xspace}
\usepackage{graphicx}
\usepackage{a4wide}
\usepackage{color}
\usepackage{listings}

\usepackage{xcolor}
% Define softer custom colors if desired
% \definecolor{darkgray}{rgb}{0.3,0.3,0.3}
% \definecolor{teal}{rgb}{0.0,0.4,0.4}
% \definecolor{navy}{rgb}{0.0,0.1,0.3}

%\usepackage{hyperref}

\usepackage{subcaption}
\usepackage{xstring}
\newcommand{\escapeunderscores}[1]{%
  \StrSubstitute{#1}{_}{\_}%
}
%\usepackage[style=numeric,citestyle=numeric-comp,sorting=ynt,defernumbers=true,backend=biber]{biblatex}

% \textwidth 21cm
% \linewidth 21cm
% \columnwidth 21cm
% 
%\marginparwidth 0 in

%\textheight 22.5cm \textwidth 16cm
%\oddsidemargin 0.0cm \evensidemargin 0.0cm
%\topmargin -2.3cm  % for nikhef
%\topmargin -0.5cm  % for hep-ph

\newcommand{\repolink}[2]{\href{https://github.com/hoppet-code/hoppet/blob/master/#1}{\ttt{#2}}}
\newcommand{\masterlink}[1]{\repolink{#1}{\escapeunderscores{#1}}}


\def\APFELPP{{APFEL\nolinebreak[4]\hspace{-.05em}\raisebox{.4ex}{\tiny\textbf{++}}}}
\newcommand{\CPP}{{C\nolinebreak[4]\hspace{-.05em}\raisebox{.4ex}{\tiny\textbf{++}}}\xspace}
\newcommand{\be}{\begin{equation}}
\newcommand{\ee}{\end{equation}}
\newcommand{\bea}{\begin{eqnarray}}
\newcommand{\eea}{\end{eqnarray}}
\newcommand{\bi}{\begin{itemize}}
\newcommand{\ei}{\end{itemize}}
\newcommand{\ben}{\begin{enumerate}}
\newcommand{\een}{\end{enumerate}}
\newcommand{\la}{\left\langle}
\newcommand{\ra}{\right\rangle}
\newcommand{\lc}{\left[}
\newcommand{\rc}{\right]}
\newcommand{\lp}{\left(}
\newcommand{\rp}{\right)}
\newcommand{\aq}{\alpha_s\left( Q^2 \right)}
\newcommand{\amz}{\alpha_s\left( M_Z^2 \right)}
\newcommand{\aqq}{\alpha_s \left( Q^2_0 \right)}
\newcommand{\aqz}{\alpha_s \left( Q^2_0 \right)}
\newcommand{\nf}{n_f)}
\newcommand{\nn}{\nonumber}
\newcommand{\nin}{\noindent}

\newcommand{\dy}{\ttt{dy}}
\newcommand{\dlnlnQ}{\ttt{dlnlnQ}}
\newcommand{\bq}{\boldsymbol{q}}
\newcommand{\mus}{\;\mu\mathrm{s}}
\newcommand{\ms}{\;\mathrm{ms}}
\newcommand{\GeV}{\;\mathrm{GeV}}
\newcommand{\TeV}{\;\mathrm{TeV}}
\newcommand{\ns}{\;\mathrm{ns}}
\newcommand{\as}{\alpha_s}

\newcommand{\ntlo}{N$^3$LO\xspace}

% author comments
\definecolor{darkgreen}{rgb}{0,0.6,0}
\newcommand{\comment}[1]{{\color{red}\textbf{[#1]}}}
\newcommand{\commentgz}[1]{{\color{red} [\it GZ:  #1}]}
\newcommand{\commentpn}[1]{{\color{magenta} [\it PN:  #1}]}
\newcommand{\gps}[1]{{\textcolor{darkgreen}{\comment{#1}$_\text{GPS}$}}}
\newcommand{\ak}[1]{{\textcolor{blue}{\comment{#1}$_\text{AK}$}}}


\newcommand{\eg}{e.g.\ }
\newcommand{\ie}{i.e.\ }
\newcommand{\cf}{cf.\ }
\newcommand{\MSbar}{\overline{\mathrm{MS}}}
\newcommand{\hoppet}{\textsc{hoppet}\xspace}
\newcommand{\ttt}[1]{\texttt{#1}}
\newcommand{\order}[1]{{\cal O}\left(#1\right)}
\newcommand{\fn}{\scriptsize}

\newcommand{\AllDGLAP}{Botje,Schoeffel:1998tz,Pegasus,Pascaud:2001bi,Weinzierl:2002mv,Cafarella:2003jr,Cafarella:2005zj,Cafarella:2008du,GuzziThesis,nnpdf,Kosower:1997hg,Ratcliffe:2000kp}

\newcommand{\myparagraph}[1]{\paragraph{#1}}

% Define special colors
\definecolor{comment}{rgb}{0,0.3,0}
\definecolor{identifier}{rgb}{0.0,0,0.3}


%% \lstset{% general command to set parameter(s)
%%   columns=fullflexible
%%   basicstyle=\tt, % print whole listing small
%%   keywordstyle=\color{black},
%%   % underlined bold black keywords
%%   identifierstyle=, % nothing happens
%%   commentstyle=\color{comment}, % white comments
%%   stringstyle=\ttfamily, % typewriter type for strings
%%   showstringspaces=false} % no special string spaces

\lstset{language=Fortran}
\lstset{
  columns=flexible,
  basicstyle=\tt\footnotesize,
  keywordstyle=,
  identifierstyle=\color{black},
  commentstyle=\tt\color{comment},
  mathescape=true,
  escapebegin=\color{comment},
  showstringspaces=false,
  keepspaces=true
}


\usepackage{hyperref}
\hypersetup{
  colorlinks = true,
  citecolor = teal,
  }
%   linkcolor  = darkgray,
%   citecolor  = teal,
%   urlcolor   = red,
% }
%\urlstyle{same}


\geometry{margin=25mm,
    headheight=110pt,
    footskip=30pt
}
\biboptions{sort&compress}

% We need slightly different text in the manual and in the release note.
\newif\ifreleasenote
\releasenotetrue % comment out to hide answers

%\journal{European Physical Journal C/ SciPost?}

\begin{document}
\begin{frontmatter}
\begin{flushright}
  CERN-TH-2023-237\\
  MPP-2023-285\\
  OUTP-23-15P
\end{flushright}
\title{\hoppet{} {\tt v2} release note}

\author[1]{Alexander Karlberg}\ead{alexander.karlberg@cern.ch}
\author[2]{Paolo Nason}\ead{paolo.nason@mib.infn.it}
\author[3,4]{Gavin Salam}\ead{gavin.salam@physics.ox.ac.uk}
\author[5,6]{Giulia Zanderighi}\ead{zanderi@mpp.mpg.de}
\author[7]{Fr\'ed\'eric Dreyer}\ead{dreyer.frederic@gene.com}

\affiliation[1]{organization={CERN, Theoretical Physics Department}, postcode={CH-1211} ,city={Geneva 23}, country={Switzerland}}
\affiliation[2]{organization={INFN, Sezione di Milano-Bicocca, and Universita di Milano-Bicocca, \mbox{Piazza della Scienza 3}}, postcode={20126} ,city={Milano}, country={Italy}}
\affiliation[3]{organization={Rudolf Peierls Centre for Theoretical Physics, Clarendon Laboratory, Parks Road}, postcode={OX1 3PU} ,city={Oxford}, country={UK}}
\affiliation[4]{organization={All Souls College}, postcode={OX1 4AL} ,city={Oxford}, country={UK}}
\affiliation[5]{organization={Max-Planck-Institut fur Physik, Boltzmannstr. 8}, postcode={85748} ,city={Garching}, country={Germany}}
\affiliation[6]{organization={Physik-Department, Technische Universitat Munchen, James-Franck-Strasse 1}, postcode={85748} ,city={Garching}, country={Germany}}
\affiliation[7]{organization={Prescient Design, Genentech, 149 5th Avenue},city={New York}, postcode={NY 10010}, country={USA}}

\begin{abstract}
  We document the three main new features in the v2 release series of the
  \hoppet parton distribution function evolution code, specifically
  support for N$^3$LO QCD evolution in the variable flavour number
  scheme, for the determination of hadronic structure functions for
  massless quarks up to N$^3$LO, and for QED evolution to an accuracy
  phenomenologically equivalent to NNLO QCD.
  %
  Additionally we describe a new Python interface, CMake build option,
  functionality to save a \hoppet table as an LHAPDF grid and update
  our performance benchmarks, including optimisations in
  interpolating PDF tables.
\end{abstract}

\begin{keyword}
  Perturbative QCD \sep DIS \sep DGLAP \sep QED
%% keywords here, in the form: keyword \sep keyword

%% PACS codes here, in the form: \PACS code \sep code

%% MSC codes here, in the form: \MSC code \sep code
%% or \MSC[2008] code \sep code (2000 is the default)

\end{keyword}
  \begin{textblock*}{2cm}(0.11\textwidth,0.096\textheight)  % (width)(x,y position)
    %\includegraphics[width=3cm]{logo/2025-09-logo-cropped.pdf}
    \includegraphics[width=2cm]{logo/2025-09-logo.png}
  \end{textblock*}
\end{frontmatter}
\newpage
\tableofcontents
\section{Introduction}


\hoppet~\cite{Salam:2008qg} is a parton distribution function (PDF)
evolution code written in modern Fortran,  with interfaces also for C/\CPP{} and
earlier dialects of Fortran.
%
It offers both a high-level PDF evolution interface and user-access to
lower-level functionality for operations such as convolutions of
coefficient functions and PDFs.
%
It is designed to provide flexible, fast and accurate evolution.

Since the first major release of \hoppet, the landscape of PDF
evolution codes has evolved substantially, with a range of new
open-source codes having been developed, for example the
APFEL~\cite{Bertone:2013vaa}, \APFELPP{}~\cite{Bertone:2017gds} and
EKO~\cite{Candido:2022tld} codes supplementing earlier widely-used
public codes such as QCDNUM~\cite{Botje} and
PEGASUS~\cite{Pegasus}.\footnote{Other codes such as 
  ChiliPDF~\cite{Diehl:2021gvs} appear to not yet be public.}
%
Nevertheless, \hoppet remains a powerful tool, notably for its ability
to reach high and quantifiable accuracies with competitive speed.
%
As a result it often provides a critical reference for benchmark
studies, as those presented e.g.\ in
Refs.~\cite{Dittmar:2005ed,Bertone:2024dpm}.
%
It also provides the option of fast (millisecond-scale) evolution with
accuracies $\sim 10^{-4}$, which is more than sufficient for
phenomenological applications.
%
%
Furthermore \hoppet's exposition not just of full PDF
evolution (as exploited in Refs.~\cite{Lai:2010vv,Gao:2013xoa,Butterworth:2015oua,Hou:2019efy,PDF4LHCWorkingGroup:2022cjn}),
but also of low-level functionality though a stable, public API, has
led to its use for a range of
fixed-order~\cite{Caola:2019nzf,Asteriadis:2019dte,Bargiela:2022dla},
all-order~\cite{Banfi:2010xy,Dasgupta:2014yra,Banfi:2015pju,Monni:2016ktx,Bizon:2017rah,Buonocore:2024xmy}
and Monte
Carlo~\cite{Monni:2019whf,vanBeekveld:2023ivn,Buonocore:2024pdv,vanBeekveld:2025lpz}
applications.
%
With the advent of ever more precise data from the LHC at CERN, and
future very high precision data at the forthcoming Electron-Ion
Collider~\cite{AbdulKhalek:2021gbh} at Brookhaven National Laboratory,
the continued development of tools such as \hoppet remains important
for the field.

This release note documents three major additions to \hoppet, made
available as part of release~2.0.0: 
%
(1) support for QCD evolution at N$^3$LO in the (zero mass) variable
flavour number scheme (VFNS)~\cite{Buza:1996wv} (Section~\ref{sec:n3lo-evolution});
%
(2) support for the
determination of massless hadronic structure functions, as initially developed
for calculations of vector-boson fusion, and later deep inelastic scattering, cross
sections~\cite{Cacciari:2015jma,Dreyer:2016oyx,Dreyer:2018qbw,Dreyer:2018rfu,Karlberg:2024hnl}
(Section~\ref{sec:structure-funcs});
%
(3) support for QED evolution (Section~\ref{sec:qed-evolution}),
originally developed as part of the LuxQED project for the evaluation
of the photon density inside a proton and its extension to lepton
distributions in the
proton~\cite{Manohar:2016nzj,Manohar:2017eqh,Buonocore:2020nai,Buonocore:2021bsf}.

This release also includes a range of other additions relative
to the original 1.1.0 release documented in \cite{Salam:2008qg}, in
particular a Python interface
(Section~\ref{sec:pyinterface}), a CMake-based build system
(Section~\ref{sec:cmake}), and the ability to write LHAPDF grids
(Section~\ref{sec:lhapdf}).
%
We have also updated the performance benchmarks and improved the speed
for interpolating PDF tables also when loaded from LHAPDF (Section~\ref{sec:v2-performance}).
%
\hoppet can be obtained by executing
\begin{lstlisting}
  git clone https://github.com/hoppet-code/hoppet.git
  cd hoppet
  git switch -d hoppet-2.x.y  # to switch to a specific release tag, e.g. hoppet-2.1.0
\end{lstlisting}
Unified documentation of the whole \hoppet package is part of the
distribution at \url{https://github.com/hoppet-code/hoppet} in the
\masterlink{docs/manual} directory.
%
Details of the other changes since release 1.1.0 can be found in the
\masterlink{NEWS.md} and \masterlink{ChangeLog} files from the
repository.
%
\medskip

\noindent \fbox{\parbox{\textwidth}{
\textbf{Note that as of the 2.0.0 release, \hoppet's library name has been renamed
from \texttt{hoppet\_v1} to \texttt{hoppet}, and similarly for the main
module and \CPP include file.
%
Users with an existing installation of \hoppet (v1) should make sure
that they link with the new library name.
%
Apart from this, v2 remains fully backwards compatible with code
designed for v1.
}
}}

%======================================================================
\section{Perturbative evolution in QCD}
\label{sec:pqcd}

\ifreleasenote%
  \textbf{Note:} this section is largely a repetition of Section~2 of
  Ref.~\cite{Salam:2008qg}, and is included again here to help provide
  context for the discussion that follows in subsequent
  sections.\medskip
%
\fi

First of all we set up the notation and
conventions that are used throughout \hoppet. The DGLAP
equation for a full flavour set of parton distributions, $q$, reads
\begin{equation}
  \label{eq:dglap-ns}
  \frac{\partial q(x,Q^2)}{\partial \ln Q^2} = 
  \int_x^1 \frac{dz}{z}
  P\left(z,\aq \right) q\lp \frac{x}{z},Q^2\rp \equiv 
  P\left(x,\aq\right) \otimes q\lp x,Q^2\rp,
\end{equation}
where $P$ is a matrix in flavour space and is expressed as an
expansion in powers of the strong coupling $\alpha_s$
\begin{equation}
  \label{eq:dpdf}
  P\left( z,\aq \right)
  =
  \sum_{n=1}   \lp \frac{\aq}{2\pi} \rp^n P^{(n)}(z)
  %
  %\frac{\aq}{2\pi} P^{(1)}(z)
  % +
  % +
  % \lp \frac{\aq}{2\pi} \rp^3 P^{(3)}(z) 
  % +
  % \lp \frac{\aq}{2\pi} \rp^4 P^{(4)}(z) \ ,
\end{equation}
%
The splitting functions in eq.~(\ref{eq:dglap-ns}) are known exactly
up to NNLO ($n\le 3$) in the unpolarised case
\cite{Furmanski:1980cm,Curci:1980uw,NNLO-NS,NNLO-singlet}, and
approximately at
N$^3$LO ($n\le 4$)~\cite{Gracey:1994nn,Davies:2016jie,Moch:2017uml,Gehrmann:2023cqm,Falcioni:2023tzp,Gehrmann:2023iah,McGowan:2022nag,NNPDF:2024nan,Moch:2021qrk,Falcioni:2023luc,Falcioni:2023vqq,Moch:2023tdj,Falcioni:2024xyt,Falcioni:2024qpd,Falcioni:2025hfz};
and up to NNLO
\cite{Mertig:1995ny,Vogelsang:1996im,Moch:2014sna,Moch:2015usa,Blumlein:2021enk,Blumlein:2021ryt}
in the polarised case.  The generalisation to the singlet case is
straightforward, as it is
%the generalisation of eq.~(\ref{eq:dglap-ns}) 
to the case of time-like evolution,\footnote{
The general structure of the relation between space-like
and time-like evolution and splitting functions
 has been investigated in
 \cite{Furmanski:1980cm,Curci:1980uw,Stratmann:1996hn,Dokshitzer:2005bf,Mitov:2006ic,Basso:2006nk,Dokshitzer:2006nm,Beccaria:2007bb}.
 %
 See references to those articles for more recent updates.
}
relevant for example for fragmentation function analysis,
where NNLO results
are also available \cite{Mitov:2006ic,Moch:2007tx,Almasy:2011eq}.

In terms of the flavour structure of Eq.~(\ref{eq:dglap-ns}), there
are different possible flavour bases for representing both the parton
distributions $q$ and the splitting functions.
%
Using $q_i$ to denote the $i^\text{th}$ quark, $i=1,\ldots,6$, $\bar
q_i$ for the corresponding anti-quark, and $g$ for the gluon, it is
common to use a basis expressed in terms of flavour and anti-flavour
combinations,
\begin{subequations}
  \label{eq:flavour-basis}
  \begin{align}
    q_{i}^\pm            &= q_i \pm \bar q_i\,, \\
    \intertext{and from those, non-singlet (NS), singlet ($\Sigma$) and valence (V) combinations~\cite{vanNeerven:1999ca,vanNeerven:2000uj},}
    q_{\NS,ik}^\pm &= q_i^\pm - q_k^\pm\,, \\
    q^V_\NS        &= \sum_{i=1}^{n_f} q_i^-\,,\\
    \Sigma               &= \sum_{i=1}^{n_f} q_i^+\,,
  \end{align}
\end{subequations}
with the evolution then taking the form
\begin{subequations}\label{eq:P-flavour-basis}
  \begin{alignat}{2}
    &\partial_{\ln Q^2} \, q_{\NS,ik}^\pm &&= P_\NS^\pm \otimes q_{\NS,ik}^\pm\,, \\
    &\partial_{\ln Q^2} \, q_{\NS}^V      &&= P_\NS^V   \otimes q_{\NS}^V\,, \\
    &\partial_{\ln Q^2} \, \Sigma         &&= P_{qq}    \otimes \Sigma + P_{qg} \otimes g\,\\
    &\partial_{\ln Q^2} \, g              &&= P_{gq}    \otimes \Sigma + P_{gg} \otimes g\,.
  \end{alignat}
\end{subequations}
The QCD coupling also satisfies a renormalisation group equation,
\begin{equation}
  \label{eq:as-ev}
  \frac{d\as}{d\ln Q^2} = \beta\lp \aq\rp = -\as (\beta_0\as +
  \beta_1\as^2 + 
  \beta_2\as^3 + 
  \beta_3\as^4) \,.
\end{equation}
Generally, our convention in \hoppet will be that perturbative quantities 
are defined to be coefficients of powers of $\as/2\pi$, but
note the different convention specifically in the case of the
$\beta$-function in Eq.~(\ref{eq:as-ev}).
%
This is for historical reasons and has been maintained because of the
significant number of external codes that rely on that convention.

% As with the splitting functions, most perturbative quantities in
% \hoppet are defined to be coefficients of powers of $\as/2\pi$.
% %
% However there are some places where a different convention is used,
% either for historical reasons or because external code uses a
% different convention.
% %
% In particular the $\beta$-function coefficients of the running
% coupling equation,
% \begin{equation}
%   \label{eq:as-ev}
%   \frac{d\as}{d\ln Q^2} = \beta\lp \aq\rp = -\as (\beta_0\as +
%   \beta_1\as^2 + 
%   \beta_2\as^3 + 
%   \beta_3\as^4) \,,
% \end{equation}
% are defined internally in \hoppet as multiplying powers of $\alpha_s$
% directly.

The evolution of the strong coupling and the parton distributions can
be performed in both the fixed flavour-number scheme (FFNS) and the 
variable flavour-number scheme (VFNS). In the VFNS case we 
need the matching conditions between the effective
theories with $n_f$ and $n_{f}+1$ light flavours for both the strong 
coupling $\aq$ and the parton distributions at the heavy quark
mass threshold $m_h^2$.

These matching conditions for the parton distributions
receive non-trivial contributions at higher orders.
%
To understand the flavour structure of the matching conditions, it is
useful to introduce $q_{i\le n_\ell}^{(n_f)}$ to refer to the gluon
and quark flavours with $i\le n_\ell$ in a effective theory with $n_f$
flavours.
%
Then in going from $n_\ell \to n_h = n_\ell+1$ flavours, the matching
condition takes the form
\begin{align}
  \label{eq:lp-nf1}
  q_{i\le n_\ell}^{(n_h)} = q_{i\le n_\ell}^{(n_\ell)} + A_{i\le n_\ell} \otimes q_{i\le n_\ell}^{(n_h)}
\end{align}
for the light flavours and the gluon, where the light-flavour part of
the matching matrix, $A_{i\le n_\ell}$, has the
same flavour entries, and acts in the same way as an $n_\ell$-flavour
splitting matrix, cf.\ Eq.~(\ref{eq:P-flavour-basis}).
%
Additionally the $\pm$ combinations for heavy-flavour $h$ in the
$n_h$-flavour effective theory, $q_h^{\pm(n_h)}$, are given by
\begin{subequations}
  \label{eq:hp-nf1}
  \begin{align}
    q_h^{+(n_h)} &= A_{h\Sigma} \otimes \Sigma^{(n_\ell)} + A_{hg} \otimes g^{(n_\ell)} \\
    q_h^{-(n_h)} &= A_{hV     } \otimes q_\NS^{V(n_\ell)}
  \end{align}
\end{subequations}
As with the splitting functions, the matching matrices are calculated
as an expansion in powers of the strong coupling.
%
In the $\MSbar$ (factorisation) scheme, carrying out the matching at a
scale equal to the heavy-quark mass, the series starts at order
$\alpha_s^2$~\cite{NNLO-MTM}, which in terms of the logarithmic accuracy is equivalent
to NNLO in the evolution.
%
The matching matrices are currently known up to
\ntlo~\cite{Bierenbaum:2009mv,Ablinger:2010ty,Kawamura:2012cr,Blumlein:2012vq,ABLINGER2014263,Ablinger:2014nga,Ablinger:2014vwa,Behring:2014eya,Ablinger:2019etw,Behring:2021asx,Fael:2022miw,Ablinger:2023ahe,Ablinger:2024xtt,BlumleinCode}.
%
The exact functional form depends on the choice of scheme for the
mass, e.g.\ pole v.\ $\MSbar$ mass.\footnote{In a general
  factorisation scheme, and when matching at scales other than the
  heavy-quark mass, the matching terms start at order $\as$.}
%
Note that at lower orders some of the flavour structures are trivially
equal to others, or are zero (e.g.\ $A_{hV}$ is zero at NNLO).

As is evident from Eqs.~(\ref{eq:lp-nf1}) and (\ref{eq:hp-nf1}), the matching
conditions lead to small discontinuities of the PDFs between the
$n_\ell$ and $n_h$-flavour effective theories.
%
Similarly, from NNLO onwards, the heavy-quark PDFs start from a
non-zero value at threshold, which sometimes can even be negative.
%
Physical predictions intended to be used at scales close to a heavy
quark mass and that take into account full quark-mass effects
generally include terms that compensate for these effects as one
crosses the mass threshold.
%
Most DGLAP evolution codes take the approximation that different heavy
quarks are well separated in mass.
%
However if one relaxes this assumption, then there are additional
terms that correlate nearby thresholds of different heavy
quarks~\cite{Ablinger:2025nnq}.
%
These are not yet included in \hoppet.


% its
% evolution in $Q^2$, which are cancelled by similar matching terms in
% the coefficient functions in massive VFN schemes, resulting in continuous physical
% observables. In particular, the heavy-quark PDFs start from a non-zero
% value at threshold at NNLO, which sometimes can even be negative.
% 
% for light quarks $q_{l,i}$ of flavour $i$ (quarks that are considered
% massless below the heavy quark mass threshold $m_h^2$) the matching
% between their values in the $n_f$ and $n_f+1$ effective theories
% reads:\footnote{Note that the literature focuses on the
%   $q_{l,i}+q_{l,-i}$ combination. We thank Johannes Bl\"umlein for
%   having provided the $A^{\rm NS,-}_{qq,h}(x)$ code needed for the
%   $q_{l,i}-q_{l,-i}$ combination.}:
% %\begin{equation}
% %\label{eq:lp-nf1}
% %  q_{l,i}^{\,(n_f+1)}(x,m_h^2) \: = \:  q_{l,i}^{\,(\nf}(x,m_h^2) +
% %\lp \frac{\alpha_s(m_h^2)}{2\pi} \rp^2
% %   A^{\rm ns,(2)}_{qq,h}(x) \otimes
% %  q_{l,i}^{\, (\nf}(x,m_h^2) \ ,
% %\end{equation}
% \begin{align}
% \label{eq:lp-nf1}
%   q_{l,i}^{\,(n_f+1)}(x,m_h^2) + q_{l,-i}^{\,(n_f+1)}(x,m_h^2)  & =   q_{l,i}^{\,(\nf}(x,m_h^2) + q_{l,-i}^{\,(\nf}(x,m_h^2) \notag \\ &+
%    A^{\rm NS,+}_{qq,h}(x) \otimes \left(
%    q_{l,i}^{\, (\nf}(x,m_h^2) + q_{l,-i}^{\, (\nf}(x,m_h^2)\right) \notag\\
%    & + \frac{1}{n_f} \Big\{A^{\rm PS}_{qq,h}(x) \otimes \Sigma^{\, (\nf}(x,m_h^2) \notag\\
%    & + A^{\rm S}_{qg,h}(x) \otimes g^{\, (\nf}(x,m_h^2)\Big\} \ , \notag \\
%   q_{l,i}^{\,(n_f+1)}(x,m_h^2) - q_{l,-i}^{\,(n_f+1)}(x,m_h^2)  & =   q_{l,i}^{\,(\nf}(x,m_h^2) - q_{l,-i}^{\,(\nf}(x,m_h^2) \notag \\ &+
%    A^{\rm NS,-}_{qq,h}(x) \otimes \left(
%    q_{l,i}^{\, (\nf}(x,m_h^2) - q_{l,-i}^{\, (\nf}(x,m_h^2)\right) \ ,
% \end{align}
% where $i = 1,\ldots n_f$, while for the gluon distribution, the heavy
% quark PDF $q_h$, and the singlet PDF $\Sigma(x,Q^2)$ (defined in
% Table~\ref{eq:diag_split}) one has :
% \begin{align}
% \label{eq:hp-nf1}
%   g^{(n_f+1)}(x,m_h^2)  &=
%     g^{\, (\nf}(x,m_h^2) +
%     A_{\rm gq,h}^{\rm S}(x) \otimes \Sigma^{(\nf}(x,m_h^2) +
%     A_{\rm gg,h}^{\rm S}(x) \otimes g^{(\nf}(x,m_h^2) \ ,
%   \nn \\[0.3cm]
%   (q_h+\bar{q}_{h})^{(n_f+1)}(x,m_h^2)  &=
%   A_{\rm hq}^{\rm S}(x)\otimes \Sigma^{(\nf}(x,m_h^2) 
%   + A_{\rm hg}^{\rm S}(x)\otimes g^{(\nf}(x,m_h^2)\ ,  \nn \\[0.3cm]
%   \Sigma^{(n_f+1)}(x,m_h^2)  &= \Sigma^{\, (\nf}(x,m_h^2) + \left[ A^{\rm NS,+}_{qq,h}(x) + A^{\rm PS}_{qq,h}(x) + A_{\rm hq}^{\rm S}(x)\right] \otimes \Sigma^{(\nf}(x,m_h^2) \nn \\
%   & + \left[ A^{\rm S}_{qg,h}(x) + A_{\rm hg}^{\rm S}(x) \right] \otimes g^{(\nf}(x,m_h^2)
% \end{align}
% with $q_h=\bar{q}_h$. Up to N$^3$LO the matching coefficients have the
% following expansions in $\alpha_s$
% \begin{align}
%   A^{\rm NS,\pm}_{qq,h}(x) & = \lp \frac{\alpha_s(m_h^2)}{2\pi} \rp^2
%   A^{\rm NS,\pm,(2)}_{qq,h}(x) + \lp \frac{\alpha_s(m_h^2)}{2\pi} \rp^3
%   A^{\rm NS,\pm,(3)}_{qq,h}(x) \ , \nn \\
%   A^{\rm S}_{gk,h}(x) & = \lp \frac{\alpha_s(m_h^2)}{2\pi} \rp^2
%   A^{\rm S,(2)}_{gk,h}(x) + \lp \frac{\alpha_s(m_h^2)}{2\pi} \rp^3
%   A^{\rm S,(3)}_{gk,h}(x), \quad k=q,g \ , \nn \\
%   %A^{\rm S}_{gg,h}(x) & = \lp \frac{\alpha_s(m_h^2)}{2\pi} \rp^2
%   %A^{\rm S,(2)}_{gg,h}(x) + \lp \frac{\alpha_s(m_h^2)}{2\pi} \rp^3
%   %A^{\rm S,(3)}_{gg,h}(x) \ , \nn \\
%   A^{\rm S}_{hk}(x) & = \lp \frac{\alpha_s(m_h^2)}{2\pi} \rp^2
%   A^{\rm S,(2)}_{hk}(x) + \lp \frac{\alpha_s(m_h^2)}{2\pi} \rp^3
%   A^{\rm S,(3)}_{hk}(x), \quad k=q,g \ , \nn \\
%   %A^{\rm S}_{hg}(x) & = \lp \frac{\alpha_s(m_h^2)}{2\pi} \rp^2
%   %A^{\rm S,(2)}_{hg}(x) + \lp \frac{\alpha_s(m_h^2)}{2\pi} \rp^3
%   %A^{\rm S,(3)}_{hg}(x) \ , \nn \\
%   A^{\rm PS}_{qq,h}(x) & = \lp \frac{\alpha_s(m_h^2)}{2\pi} \rp^3
%   A^{\rm PS,(3)}_{qq,h}(x) \, , \nn \\
%   A^{\rm S}_{qg,h}(x) & = \lp \frac{\alpha_s(m_h^2)}{2\pi} \rp^3
%   A^{\rm S,(3)}_{qg,h}(x)\,.
% \end{align}
% At $\mathcal{O}(\alpha_S^2)$ we have that $A^{\rm NS,+}_{qq,h}(x) =
% A^{\rm NS,-}_{qq,h}(x)$ whereas they start to differ at
% $\mathcal{O}(\alpha_S^3)$. The NNLO matching coefficients were
% computed in \cite{NNLO-MTM}\footnote{We thank the authors for the
% code corresponding to the calculation.} and the N$^3$LO matching
% coefficients
% in~\cite{Bierenbaum:2009mv,Ablinger:2010ty,Kawamura:2012cr,Blumlein:2012vq,ABLINGER2014263,Ablinger:2014nga,Ablinger:2014vwa,Behring:2014eya,Ablinger:2019etw,Behring:2021asx,Fael:2022miw,Ablinger:2023ahe,Ablinger:2024xtt,BlumleinCode}.\footnote{We
% thank Johannes Bl\"umlein for sharing a pre-release version the code from
% Ref.~\cite{BlumleinCode} with us, which also contains code associated
% with Refs.~\cite{Ablinger:2024xtt,Fael:2022miw}.
% %
% }
% %
% Notice that the above
% conditions will lead to small discontinuities of the PDFs in its
% evolution in $Q^2$, which are cancelled by similar matching terms in
% the coefficient functions in massive VFN schemes, resulting in continuous physical
% observables. In particular, the heavy-quark PDFs start from a non-zero
% value at threshold at NNLO, which sometimes can even be negative.

The corresponding N$^3$LO relation for the matching of the $\MSbar$
coupling constant at the heavy quark threshold $m^2_h$ is given by 
\begin{equation}
\label{eq:as-nf1}
  \as^{\, (n_f+1)}(m_h^2) \: = \:
  \as^{\, (\nf} (m_h^2) +   C_2 \lp \frac{\as^{\, (\nf} (m_h^2)}{2\pi} \rp^3+   C_3 \lp \frac{\as^{\, (\nf} (m_h^2)}{2\pi} \rp^4
   \:\: ,
\end{equation}
where the matching coefficients $C_2$ and $C_3$ were computed in
\cite{Chetyrkin:1997sg,Chetyrkin:1997un}.
%
The value and the form of the matching coefficients in
eqs.~(\ref{eq:lp-nf1},\ref{eq:hp-nf1}) depend on the scheme used for
the quark masses; by default in \hoppet quark masses are taken to be
pole masses, though the option exists for the user to supply and have
thresholds crossed at $\MSbar$ masses, but only up to NNLO. We note
that in the current implementation in \hoppet, the matching can only be
performed at the matching point that corresponds to the heavy-quark masses
themselves.
%
As for the PDFs, when two heavy-quark thresholds are considered close,
there are additional terms from the simultaneous decoupling of the two
heavy quarks~\cite{Grozin:2011nk}, which are not yet included in
\hoppet. 

Both evolution and threshold matching preserve the momentum sum rule
\begin{equation}
  \int_0^1 dx~x \lp \Sigma(x,Q^2)+g(x,Q^2)\rp =1 \,,
\end{equation}
and valence sum rules
\begin{equation}
  \int_0^1 dx\, \left[q(x,Q^2)-{\bar q}(x,Q^2) \right] = \left\{ 
    \begin{array}{ll}
      1, & \text{for } q = d \text{ (in proton)}\\
      2, & \text{for } q = u \text{ (in proton)}\\
      0, & \text{other flavours}
    \end{array}
    \right.
\end{equation}
as long as they hold at the initial scale (occasionally not the case,
\eg in modified LO sets for Monte Carlo
generators~\cite{Sherstnev:2008dm}).

% The default basis for the PDFs, called the \ttt{human} 
% representation in \hoppet, is such that 
%  the entries in an array
% \ttt{pdf(-6:6)} of PDFs correspond to:
% \bea 
% \bar t={-6} \ ,  \bar b={-5} \ ,  \bar c={-4}
% \ , \nn   \bar s&=&{-3} \ , \nn  \bar u={-2} \ , \nn
%  \bar d={-1} \ , \\  g&=&{0} \ , \\ \nn   d={1} \ , \nn  u={2} 
% \ , \nn  
% s={3} \ , \nn   c&=&{4} \ , \nn b={5} \ , \nn  t={6} \ . \nn 
% \eea
% %  This representation is the
% % same as that used in the \ttt{LHAPDF} library \cite{LHAPDF}. 
% %
% However, this representation leads
% to a complicated form of the evolution equations.
% The splitting matrix can be simplified considerably (made diagonal
% except for a $2\times2$ singlet block) by switching to a different
% flavour representation, which is named
% the \ttt{evln} representation, for the PDF set, as explained in detail in
% \cite{vanNeerven:1999ca,vanNeerven:2000uj}. This representation
% is described in Table \ref{eq:diag_split}.
% 
% In the {\tt evln} basis, 
% the gluon evolves coupled to the singlet  PDF $\Sigma$,
% and all non-singlet PDFs evolve independently.
% Notice that the representations of the PDFs
% are preserved under linear operations, so in particular
% they are preserved under DGLAP evolution.
% The conversion from the \ttt{human} to the \ttt{evln}
% representations of PDFs requires that the number of
% active quark flavours $n_f$ be specified by the user, as described in
% \ifreleasenote
% Section~5.1.2 of Ref.~\cite{Salam:2008qg}.
% \else
% Section~\ref{sec:evln-rep}.
% \fi
% 
% \begin{table}
% \begin{center}
% \begin{tabular}{|r | c | l |}
% \hline
%      i & \mbox{name} & $q_i$ \\ \hline
%      $ -6\ldots-(n_f+1)$ & $q_i$ & $q_i$\\
%      $-n_f\ldots -2$ & $q_{\mathrm{NS},i}^{-}$ & 
% $(q_i -  {\bar q}_i) - (q_1 - {\bar q}_1)$\\
%       -1           & $q_{\mathrm{NS}}^{V}$ & 
% $\sum_{j=1}^{n_f} (q_j -  {\bar q}_j)$\\
%        0           & g & \textrm{gluon} \\
%        1           & $\Sigma$ & $\sum_{j=1}^{n_f} (q_j +  {\bar q}_j)$\\
%      $2\ldots n_f$ & $q_{\mathrm{NS},i}^{+}$ &
% $ (q_i +  {\bar q}_i) - (q_1 + {\bar q}_1)$\\
%       $(n_f+1)\ldots6$ & $q_i$ & $q_i$ \\
% \hline
% \end{tabular}
% \caption{}{\label{eq:diag_split} The evolution representation 
% (called \ttt{evln} in \hoppet)
% of PDFs with $n_f$ active quark flavours
% in terms of the \ttt{human} representation.}  
% \end{center}
% \end{table}
 
In \hoppet, unpolarised DGLAP evolution is available up to N$^3$LO
in the $\MSbar$ scheme, while for the DIS scheme
only evolution up to NLO is available, but without the NLO heavy-quark
threshold matching conditions. For polarised evolution up to NLO only
the $\MSbar$ scheme is available. The variable \ttt{factscheme}
takes different values for each factorisation scheme:
\begin{center}
  \begin{tabular}{|c|l|}\hline
    \ttt{factscheme} & Evolution\\[2pt]\hline
    1 & unpolarised $\MSbar$ scheme\\[2pt]\hline
    2 & unpolarised DIS scheme\\[2pt]\hline
    3 & polarised $\MSbar$ scheme\\\hline
  \end{tabular}
\end{center}
Note that mass thresholds are currently
missing in the DIS scheme.

The extension to QED is conceptually straightforward.
%
Further discussion of that is given in
Section~\ref{sec:qed-evolution}.


%%% Local Variables:
%%% TeX-master: "HOPPET-doc.tex"
%%% End:


%======================================================================
\section{Brief summary of \hoppet structure}
\label{sec:hoppet-structure}

\hoppet works in $x$-space.
%
It represents PDFs and splitting functions on grids, typically
multiple nested grids, each uniform in $y= \ln 1/x$, with the nesting
involving smaller spacings and smaller $y$ ranges so as to achieve
good accuracy not just at small $x$ but also large $x$.
%
The underlying convolutions of splitting functions with PDFs
effectively use piecewise polynomial interpolations of the PDFs.
%
The convolutions of the splitting functions with individual basis
polynomials are pre-evaluated using adaptive Gaussian integration.
%
Evolution equations are solved using Runge-Kutta methods.

\begin{table}[pt!]
  \centering
  \newcommand{\tabsecheader}[1]{\multicolumn{2}{c}{\bf #1\hspace{5.5em}\mbox{ }}}
%
\begin{tabular}{lp{0.55\textwidth}}
%\hline
%\bf METHOD  & \bf DESCRIPTION \\
%%--------------------------------------------------------
 \toprule
\tabsecheader{Configuration (optional)} \\\midrule
%--------------------------------------------------------
\begin{lstlisting}
hoppetSetExactDGLAP(threshold,split)
\end{lstlisting}
&\fn
Sets use of exact NNLO mass thresholds and splitting functions
  (default: both false).
  \\\hline
\begin{lstlisting}
hoppetSetApproximateDGLAPN3LO(variant)
hoppetSetSplittingN3LO(variant)
hoppetSetN3LOnfthresholds(variant)     
\end{lstlisting}
&\fn
  Sets variants for the choice of \ntlo splitting functions and
  thresholds, cf.\ Sec.~\ref{sec:n3lo-evolution}.
\\\hline
% 
\begin{lstlisting}
hoppetSetQED(withqed,qcdqed,plq)
\end{lstlisting}
&  \fn Sets QED evolution and its options
  (cf.~Sec.~\ref{sec:qed-evolution}; default: all false).
\\
%%--------------------------------------------------------
  %\hline\hline
  \toprule
\tabsecheader{Initialisation} \\\midrule
%--------------------------------------------------------
%\midrule
% \multicolumn{2}{c}{\bf Initialisation}  \\
%\tabsecheader{Initialisation}\\
%\multicolumn{2}{|c|}{\bf Initialisation\hspace{5.5em}\mbox{ }}  \\
%\midrule
\begin{lstlisting}
hoppetStart(dy,nloop)
\end{lstlisting} 
& \fn
Sets up a compound grid with
spacing in $\ln 1/x$ of \ttt{dy} at small $x$,
extending to $y = 12$ and numerical
order $\ttt=-6$. The $Q$ range for the tabulation will be $1\GeV <
Q<28 \TeV$, \fn~ \ttt{dlnlnQ=dy/4} and the factorisation scheme is
  ${\overline{\rm MS}}$ (\ttt{factscheme\_MSbar}).
  \\
\midrule
\begin{lstlisting}
hoppetStartExtended(ymax,dy,Qmin,
  Qmax,dlnlnQ,nloop,order,factscheme)
\end{lstlisting} & \fn
  More general initialisation. \\
\midrule 
\begin{lstlisting}
hoppetSetFFN(fixed_nf)
hoppetSetPoleMassVFN(mc,mb,mt)
hoppetSetMSbarMassVFN(mc,mb,mt)
\end{lstlisting} &
\fn Set heavy flavour scheme ($\MSbar$ available only to NNLO).\\
%            &\\[-0.5em]
%--------------------------------------------------------
\toprule
\tabsecheader{Normal evolution} \\\midrule
%--------------------------------------------------------
\begin{lstlisting}
hoppetEvolve(asQ0,Q0alphas,
           nloop,muR_Q,LHAsub,Q0pdf)
\end{lstlisting} &
\fn PDF evolution: specifies the coupling \ttt{asQ0} at a 
scale \ttt{Q0alphas}, 
% \texttt{\footnotesize nloop,muR\_Q,LHAsub,Q0pdf)}&
  the number of loops for  evol., \ttt{nloop},
  the ratio (\ttt{muR\_Q}) of ren. to fact. scales,
 the name of a subroutine \ttt{LHAsub(x,Q,f}) that fills \ttt{f(-6:6)},
 and the scale
\ttt{Q0pdf} at which one starts the PDF evolution.
 Note:  \ttt{LHAsub} is only called at scale \ttt{Q0pdf}
and in \CPP \ttt{f[iflv]} spans \ttt{iflv=0..12}.
\\
%The PDF at a given value of  is obtained \\
%(multiplied by $x$) at the given $\ttt{x}$ and $\ttt{Q}$ values \\
%\midrule&\\[-0.5em]
%--------------------------------------------------------
\toprule
\tabsecheader{Cached evolution} \\\midrule
%--------------------------------------------------------
%\bf Cached evolution & \\
\begin{lstlisting}
hoppetPreEvolve(asQ,Q0alphas, 
                nloop,muR_Q,Q0pdf)
\end{lstlisting} & \fn
 Preparation of a cached evolution.\\
\midrule
\begin{lstlisting}
hoppetCachedEvolve(LHAsub)
\end{lstlisting} &
\fn  Perform cached evolution with the initial condition
     at \ttt{Q0pdf} from a routine \ttt{LHAsub} 
with LHAPDF-like interface.
 Note: \ttt{LHAsub} only called at scale \ttt{Q0pdf}.\\
%--------------------------------------------------------
\toprule
\tabsecheader{Evaluation} \\\midrule
%--------------------------------------------------------
\begin{lstlisting}
hoppetEval(x,Q,f)
\end{lstlisting} &
\fn On return, \ttt{f(-6:6)} contains all flavours of the PDF set
                   (multiplied by $x$). In \CPP, the array
                   indices span 0 to 12. Increase upper bound by 5
                   with QED.\\
\midrule
\begin{lstlisting}
hoppetEvalSplit(x,Q,iloop,nf,pf)
\end{lstlisting} &
\fn On return, \ttt{pf(-6:6)} contains the (cached) convolution of the
                   \ttt{iloop} splitting function ($1=\text{LO}$) with the tabulated
                   PDF for the given \texttt{nf}. One can chain splitting
                   functions up to $\order{\as^4}$, e.g.\ \texttt{iloop=31} gives
                   $P_\text{NNLO}\otimes P_\text{LO} \otimes f$.\\
\midrule
\begin{lstlisting}
hoppetAlphaS(Q)
\end{lstlisting} &
\fn Returns the coupling at scale $Q$. \\  \midrule
%
\begin{lstlisting}
hoppetWriteLHAPDFGrid(basename,
                          pdf_index)
\end{lstlisting}
            & \fn
              Write an LHAPDF grid file to \ttt{basename\_nnnn.dat} where
              \ttt{nnnn} is \ttt{pdf\_index}; if
              \ttt{pdf\_index} is
              0, also write a template \ttt{basename.info} file.\\
%--------------------------------------------------------
\toprule
\tabsecheader{Cleanup (optional)} \\\midrule
%--------------------------------------------------------
\begin{lstlisting}
hoppetDeleteAll()
\end{lstlisting}
            & \fn
              Deletes all storage allocated by the streamlined interface
%--------------------------------------------------------
  \\\midrule
\end{tabular}

  \caption{Core methods of the streamlined interface in Fortran and \CPP.
    %
    In Python, \ttt{hoppetStart(...)} is to be replaced with
    \ttt{hoppet.Start(...)}, and routines like \ttt{hoppet.Eval(...)}
    and \ttt{LHAsub(x,Q)} return
    \texttt{f} rather than taking \texttt{f} as an argument and
    filling it.
    %
    \label{tab:streamlined-interface}
  }
\end{table}


The code has two interfaces.
%
For simple usage, it provides a so-called ``streamlined''
interface, giving high-level access to the functionality that is most
widely needed.
%
It is available from Fortran, \CPP and, as of v2, Python
(cf.\ Sec.~\ref{sec:pyinterface}).
%
Its main routines are listed in Table~\ref{tab:streamlined-interface}.
%
The functionality includes evolving to fill a PDF tabulation and then
accessing that PDF tabulation at given $x,Q$ points.
%
For faster tabulation of many distinct initial conditions, one can
pre-determine (cache) the evolution operators between the different
$Q$ scales at which the PDF is tabulated, and then repeatedly apply
that cached evolution.
%
The streamlined interface also provides access to convolutions of the
various orders of splitting functions with the tabulated PDF.


\begin{table}[pt!]
  \centering
  %\newcommand{\tabtt}[1]{\lstinline|#1|}
\newcommand{\tabtt}[1]{\footnotesize\texttt{#1}}
\begin{tabular}{ll}
\toprule
%\bf TYPES  & \bf DESCRIPTION \\
%\hline
\lstinline|type(grid_def :: grid| & \fn $x-$space grid definition \\ 
\midrule
\lstinline|real(dp), pointer :: gluon(:)| & \fn Holds a 
`grid quantity' (\eg gluon PDF) \\
%\hline
\lstinline|real(dp), pointer :: PDFset(:,:)| & \fn
Grid representation of a (13-flavour) PDF set \\
\midrule
\lstinline|type(grid_conv)  :: Pgg| 
 & \fn Convolution operator ({\it i.e.} splitting function) \\ 
%\hline
\lstinline|type(split_mat)  :: Pmat|
 & \fn Splitting matrix (with full flavour structure)\\ 
%\hline
 \lstinline|type(mass_threshold_mat) :: MTM|
 & \fn Heavy quark mass-threshold matrix\\ 
%\hline
\lstinline|type(dglap_holder) :: dglap_h| & \fn DGLAP holder (\ie all
splitting and mass-threshold matrices) \\
\midrule
\lstinline|type(running_coupling) :: coupling| & \fn Running coupling \\
%\hline
\lstinline|type(evln_operator) :: evop| & \fn Evolution operator (linked
list of split \&  mass-threshold matrices)\\
%\hline
\lstinline|type(pdf_table) :: table| & \fn PDF set tabulated in $x$ \& $Q$\\
\bottomrule\normalsize
\end{tabular}

  \caption{Core objects in the general interface.
    %
    \label{tab:general-interface}
  }
\end{table}


For more advanced usage there is a ``general'' interface (sometimes
called the object-oriented interface, though it is only partially so).
%
It gives access to the various low-level objects that are useful in
DGLAP evolution, such as splitting functions, splitting matrices,
tabulations of PDFs, etc.
%
The main objects are listed in Table~\ref{tab:general-interface}.
%
Up to v2.1, that interface is accessible only in modern Fortran,
though in due course we expect to extend it to other languages.
%
We refer the reader to the original manual~\cite{Salam:2008qg} for an
extended discussion of the general interface.
%
For some uses, it can be convenient to initialise \hoppet with the
streamlined interface and then access the underlying objects in
Fortran from the
\repolink{src/streamlined_interface.f90}{streamlined\_interface}
module. 

Various examples are available with the two sets of interfaces, to be
found in the \masterlink{examples/} directory of the repository.
%
This document will focus mostly on the streamlined interface, though
in a places we will also discuss key additions to the general interface.

%======================================================================
\input{v2-updates.tex}

%======================================================================
\section{Conclusion}

Version 2 of \hoppet brings major additions to its functionality.
%
These include evolution up to N$^3$LO, massless structure function
evaluation, QED evolution, a Python interface, a modern build system,
functionality for writing LHAPDF grids and significant speed
improvements in the interpolation of its internal PDF tables.

Overall \hoppet remains highly competitive in terms of speed and
accuracy.
%
For example, repeated filling of a full PDF tabulation takes about a
millisecond per initial condition, with a relative accuracy of
$10^{-4}$ or better for $10^{-5} < x<0.9$.
%
It offers explicit handles to control the accuracy, allowing users to
verify the precision of their results and choose the optimal trade-off
between speed and precision.
%
Its modern Fortran interface also offers powerful and flexible access
to a range of common PDF manipulations such as convolutions with
arbitrary splitting and coefficient functions, features that are
useful in a variety of contexts.

We hope that this release of \hoppet can help provide solid
foundations for a range of groups to contribute to the ongoing
discussions~\cite{McGowan:2022nag,NNPDF:2024nan,Cooper-Sarkar:2024crx,Cridge:2024icl,Cooper-Sarkar:2025sqw}
in the field concerning the impact of \ntlo and QED effects in PDF
fits.
%
We also hope that the interfaces across computing languages will
facilitate the practical aspects of integration with a range of other
tools.


\section*{Acknowledgements}

We are grateful to Johannes Bl\"umlein for providing us with a
pre-release version of an exact Fortran code corresponding
Ref.~\cite{BlumleinCode} as well as a suitable license for its use,
and to Arnd Behring for assistance with \ttt{libome}.
%
We also gratefully acknowledge Luca Buonocore for his implementation
of the $P_{lq}$ splitting function in the QED code, Andrii Verbitskyi
for contributing the initial version of the CMake build system,
%
and Melissa van Beekveld for collaboration on initial options for speed
improvements in the evaluation of tabulated PDFs.
%
We thank Valerio Bertone for cross-checks of the  structure
functions and PDF evolution with \APFELPP.
%
We also wish to thank Juan Rojo for useful discussions. 
%
GPS acknowledges funding from a Royal Society Research
Professorship (grant RP$\backslash$R$\backslash$231001) and from the Science and
Technology Facilities Council (STFC) under grant ST/X000761/1.
%
PN thanks the Humboldt Foundation for support. 
%
%\appendix


\bibliographystyle{elsarticle-num}
\bibliography{hoppet.bib}
%\begin{thebibliography}{99}

\bibitem{Botje}
  M.~Botje, QCDNUM, \url{http://www.nikhef.nl/~h24/qcdnum/}~.

% Uses decomposition on Laguerre polynomials -- about
% 30 of them, remains Y^2 * T method. Initialisation
% (transform of splitting functions takes 15s on thalie)
% (didn't try evolution; didn't check accuracy; evolution
% times and accuracy are not mentioned; seemed fixed nf)
\bibitem{Schoeffel:1998tz}
L.~Schoeffel,
%``An elegant and fast method to solve QCD evolution equations,  application to
%the determination of the gluon content of the pomeron,''
Nucl.\ Instrum.\ Meth.\ A {\bf 423} (1999) 439.
%%CITATION = NUIMA,A423,439;%%
See also \url{http://www.desy.de/~schoffel/L_qcd98.html},
\url{http://www-spht.cea.fr/pisp/gelis/Soft/DGLAP/index.html}

\bibitem{Pegasus}
  A.~Vogt,
  %``Efficient evolution of unpolarized and polarized parton distributions  with
  %QCD-PEGASUS,''
  Comput.\ Phys.\ Commun.\  {\bf 170} (2005) 65
  [arXiv:hep-ph/0408244].
  %%CITATION = HEP-PH 0408244;%%

\bibitem{Pascaud:2001bi}
C.~Pascaud and F.~Zomer,
%``A fast and precise method to solve the Altarelli-Parisi equations in x
%space,''
arXiv:hep-ph/0104013.
%%CITATION = HEP-PH 0104013;%%


\bibitem{Weinzierl:2002mv}
S.~Weinzierl,
%``Fast evolution of parton distributions,''
Comput.\ Phys.\ Commun.\  {\bf 148} (2002) 314
[arXiv:hep-ph/0203112];
%%CITATION = HEP-PH 0203112;%%
%\bibitem{Roth:2004ti}
M.~Roth and S.~Weinzierl,
%``QED corrections to the evolution of parton distributions,''
Phys.\ Lett.\ B {\bf 590} (2004) 190
[arXiv:hep-ph/0403200].
%%CITATION = HEP-PH 0403200;%%

% about 1 minute at NLO.
\bibitem{coriano} A.~Cafarella and C.~Coriano,
%``Direct solution of renormalization group equations of QCD in x-space: NLO
%implementations at leading twist,''
Comput.\ Phys.\ Commun.\  {\bf 160} (2004) 213
[arXiv:hep-ph/0311313];
%%CITATION = HEP-PH 0311313;%%
%\bibitem{Cafarella:2005zj}
  A.~Cafarella, C.~Coriano' and M.~Guzzi,
  %``NNLO logarithmic expansions and exact solutions of the DGLAP equations from
  %x-space: New algorithms for precision studies at the LHC,''
  Nucl.\ Phys.\  B {\bf 748} (2006) 253
  [arXiv:hep-ph/0512358];
  %%CITATION = NUPHA,B748,253;%%
  A.~Cafarella, C.~Coriano and M.~Guzzi,
  %``Precision Studies of the NNLO DGLAP Evolution at the LHC with CANDIA,''
  arXiv:0803.0462 [hep-ph].

\bibitem{GuzziThesis}
  M.~Guzzi, Ph.D. Thesis, Lecce University, 2006 [hep-ph/0612355].

\bibitem{nnpdf}
  L.~Del Debbio, S.~Forte, J.~I.~Latorre, A.~Piccione and J.~Rojo  [NNPDF
                  Collaboration],
  %``Neural network determination of parton distributions: The nonsinglet
  %case,''
  JHEP {\bf 0703} (2007) 039
  [arXiv:hep-ph/0701127].

\bibitem{Kosower:1997hg}
  D.~A.~Kosower,
  %``Evolution of parton distributions,''
  Nucl.\ Phys.\  B {\bf 506} (1997) 439
  [arXiv:hep-ph/9706213].

\bibitem{Ratcliffe:2000kp}
  P.~G.~Ratcliffe,
  %``A matrix approach to numerical solution of the DGLAP evolution
  %equations,''
  Phys.\ Rev.\  D {\bf 63}, 116004 (2001)
  [arXiv:hep-ph/0012376].
  %%CITATION = PHRVA,D63,116004;%%

\bibitem{DGLAP}
V.N.~Gribov and L.N.~Lipatov, 
%\sjnp{15}{1972}{438};
Sov.\ J.\ Nucl.\ Phys. {\bf 15} (1972) 438;
%``Deep Inelastic E P Scattering In Perturbation Theory,''
%[Sov.\ J.\ Nucl.\ Phys.\  {\bf 15} (1972) 438].
%%CITATION = YAFIA,15,781;%%
G.~Altarelli and G.~Parisi, 
%\npb{126}{1977}{298};
Nucl.\ Phys.\ B {\bf 126} (1977) 298;
%``Asymptotic Freedom In Parton Language,''
%%CITATION = NUPHA,B126,298;%%
Yu.L.~Dokshitzer, 
%\jetp{46}{1977}{641}.
Sov.\ Phys.\ JETP {\bf 46} (1977) 641.
%``Calculation Of The Structure Functions For Deep Inelastic Scattering And E+ E- Annihilation By Perturbation Theory In Quantum Chromodynamics.
%[Zh.\ Eksp.\ Teor.\ Fiz.\  {\bf 73} (1977) 1216].
%%CITATION = SPHJA,46,641;%%

\bibitem{CTEQ}
  J.~Pumplin, D.~R.~Stump, J.~Huston, H.~L.~Lai, P.~Nadolsky and W.~K.~Tung,
  %``New generation of parton distributions with uncertainties from global  QCD
  %analysis,''
  JHEP {\bf 0207}, 012 (2002)
  [arXiv:hep-ph/0201195].

%\cite{Martin:2002dr}
\bibitem{MRST}
  A.~D.~Martin, R.~G.~Roberts, W.~J.~Stirling and R.~S.~Thorne,
  %``NNLO global parton analysis,''
  Phys.\ Lett.\  B {\bf 531} (2002) 216
  [arXiv:hep-ph/0201127].
  %%CITATION = PHLTA,B531,216;%%

\bibitem{DisResum}
  M.~Dasgupta and G.~P.~Salam,
  %``Resummation of the jet broadening in DIS,''
  Eur.\ Phys.\ J.\  C {\bf 24} (2002) 213
  [arXiv:hep-ph/0110213];
  %%CITATION = EPHJA,C24,213;%%
%\bibitem{Dasgupta:2002dc}
  %M.~Dasgupta and G.~P.~Salam,
  %``Resummed event-shape variables in DIS,''
  JHEP {\bf 0208} (2002) 032
  [arXiv:hep-ph/0208073].
  %%CITATION = JHEPA,0208,032;%%



\bibitem{caesar}
  A.~Banfi, G.~P.~Salam and G.~Zanderighi,
  %``Principles of general final-state resummation and automated
  %implementation,''
  JHEP {\bf 0503}, 073 (2005)
  [arXiv:hep-ph/0407286];
  %%CITATION = JHEPA,0503,073;%%
%
%\cite{Banfi:2004nk}
%\bibitem{Banfi:2004nk}
%  A.~Banfi, G.~P.~Salam and G.~Zanderighi,
  %``Resummed event shapes at hadron hadron colliders,''
  JHEP {\bf 0408}, 062 (2004)
  [arXiv:hep-ph/0407287].
  %%CITATION = JHEPA,0408,062;%%

\bibitem{Ciafaloni:2003rd}
  M.~Ciafaloni, D.~Colferai, G.~P.~Salam and A.~M.~Stasto,
  %``Renormalisation group improved small-x Green's function,''
  Phys.\ Rev.\  D {\bf 68}, 114003 (2003)
  [arXiv:hep-ph/0307188].


\bibitem{APPL}
  T.~Carli, G.~P.~Salam and F.~Siegert,
  %``A posteriori inclusion of PDFs in NLO QCD final-state calculations,''
  :hep-ph/0510324;
  %%CITATION = HEP-PH/0510324;%%
  T.~Carli, D.~Clements, {\it et al.}, in preparation.

\bibitem{Banfi:2007gu}
  A.~Banfi, G.~P.~Salam and G.~Zanderighi,
  %``Accurate QCD predictions for heavy-quark jets at the Tevatron and LHC,''
  JHEP {\bf 0707} (2007) 026
  [arXiv:0704.2999 [hep-ph]].


\bibitem{Benchmarks} 
  W.~Giele {\it et al.},
  ``Les Houches 2001, the QCD/SM working group: Summary report,''
  hep-ph/0204316, section 1.3;\\
  %%CITATION = HEP-PH 0204316;%%
  M.~Dittmar {\it et al.},
  ``Parton distributions: Summary report for the HERA-LHC workshop,''
  hep-ph/0511119, section 4.4.
  %%CITATION = HEP-PH 0511119;%%

\bibitem{LHAPDF} W.~Giele and M.~R.~Whalley,
\url{http://hepforge.cedar.ac.uk/lhapdf/}


\bibitem{CFP}
  W.~Furmanski and R.~Petronzio,
  %``Singlet Parton Densities Beyond Leading Order,''
  Phys.\ Lett.\  B {\bf 97} (1980) 437;
  %%CITATION = PHLTA,B97,437;%%
  G.~Curci, W.~Furmanski and R.~Petronzio,
  %``Evolution Of Parton Densities Beyond Leading Order: The Nonsinglet Case,''
  Nucl.\ Phys.\  B {\bf 175} (1980) 27.
  %%CITATION = NUPHA,B175,27;%%


%\cite{Moch:2004pa}
\bibitem{NNLO-NS}
  S.~Moch, J.~A.~M.~Vermaseren and A.~Vogt,
  %``The three-loop splitting functions in QCD: The non-singlet case,''
  Nucl.\ Phys.\ B {\bf 688} (2004) 101
  [arXiv:hep-ph/0403192].
  %%CITATION = HEP-PH 0403192;%%

%\cite{Vogt:2004mw}
\bibitem{NNLO-singlet}
  A.~Vogt, S.~Moch and J.~A.~M.~Vermaseren,
  %``The three-loop splitting functions in QCD: The singlet case,''
  Nucl.\ Phys.\ B {\bf 691} (2004) 129
  [arXiv:hep-ph/0404111].
  %%CITATION = HEP-PH 0404111;%%

\bibitem{Mertig:1995ny}
  R.~Mertig and W.~L.~van Neerven,
  %``The Calculation Of The Two Loop Spin Splitting Functions P(Ij)(1)(X),''
  Z.\ Phys.\  C {\bf 70} (1996) 637
  [arXiv:hep-ph/9506451].
  %%CITATION = ZEPYA,C70,637;%%

\bibitem{Vogelsang:1996im}
  W.~Vogelsang,
  %``The spin-dependent two-loop splitting functions,''
  Nucl.\ Phys.\  B {\bf 475} (1996) 47
  [arXiv:hep-ph/9603366].
  %%CITATION = NUPHA,B475,47;%%


\bibitem{Stratmann:1996hn}
  M.~Stratmann and W.~Vogelsang,
  %``Next-to-leading order evolution of polarized and unpolarized  fragmentation
  %functions,''
  Nucl.\ Phys.\  B {\bf 496} (1997) 41
  [arXiv:hep-ph/9612250].
  %%CITATION = NUPHA,B496,41;%%

%\cite{Dokshitzer:2005bf}
\bibitem{Dokshitzer:2005bf}
  Yu.~L.~Dokshitzer, G.~Marchesini and G.~P.~Salam,
  %``Revisiting parton evolution and the large-x limit,''
  Phys.\ Lett.\  B {\bf 634}, 504 (2006)
  [arXiv:hep-ph/0511302].
  %%CITATION = PHLTA,B634,504;%%


\bibitem{Mitov:2006ic}
  A.~Mitov, S.~Moch and A.~Vogt,
  %``Next-to-next-to-leading order evolution of non-singlet fragmentation
  %functions,''
  Phys.\ Lett.\  B {\bf 638} (2006) 61
  [arXiv:hep-ph/0604053].

\bibitem{Basso:2006nk}
  B.~Basso and G.~P.~Korchemsky,
  %``Anomalous dimensions of high-spin operators beyond the leading order,''
  Nucl.\ Phys.\  B {\bf 775} (2007) 1
  [arXiv:hep-th/0612247].
  %%CITATION = NUPHA,B775,1;%%

\bibitem{Dokshitzer:2006nm}
  Yu.~L.~Dokshitzer and G.~Marchesini,
  %``N = 4 SUSY Yang-Mills: Three loops made simple(r),''
  Phys.\ Lett.\  B {\bf 646} (2007) 189
  [arXiv:hep-th/0612248].

\bibitem{Beccaria:2007bb}
  M.~Beccaria, Yu.~L.~Dokshitzer and G.~Marchesini,
  %``Twist 3 of the sl(2) sector of N=4 SYM and reciprocity respecting
  %evolution,''
  Phys.\ Lett.\  B {\bf 652} (2007) 194
  [arXiv:0705.2639 [hep-th]].



\bibitem{NNLO-MTM}
  M.~Buza, Y.~Matiounine, J.~Smith, R.~Migneron and W.~L.~van Neerven,
  %``Heavy quark coefficient functions at asymptotic values $Q~2 \gg m~2$,''
  Nucl.\ Phys.\ B {\bf 472}, 611 (1996)
  [arXiv:hep-ph/9601302];\\
  %%CITATION = HEP-PH 9601302;%%
%
  M.~Buza, Y.~Matiounine, J.~Smith and W.~L.~van Neerven,
  %``Charm electroproduction viewed in the variable-flavour number scheme
  %versus fixed-order perturbation theory,''
  Eur.\ Phys.\ J.\ C {\bf 1}, 301 (1998)
  [arXiv:hep-ph/9612398].
  %%CITATION = HEP-PH 9612398;%%

\bibitem{Chetyrkin:1997sg}
  K.~G.~Chetyrkin, B.~A.~Kniehl and M.~Steinhauser,
  %``Strong coupling constant with flavour thresholds at four loops in the
  %MS-bar scheme,''
  Phys.\ Rev.\ Lett.\  {\bf 79}, 2184 (1997)
  [arXiv:hep-ph/9706430].

%\cite{Sherstnev:2008dm}
\bibitem{Sherstnev:2008dm}
  A.~Sherstnev and R.~S.~Thorne,
  %``Different PDF approximations useful for LO Monte Carlo generators,''
  arXiv:0807.2132 [hep-ph].
  %%CITATION = ARXIV:0807.2132;%%

\bibitem{vanNeerven:1999ca}
  W.~L.~van Neerven and A.~Vogt,
  %``NNLO evolution of deep-inelastic structure functions: The non-singlet
  %case,''
  Nucl.\ Phys.\ B {\bf 568} (2000) 263
  [arXiv:hep-ph/9907472].
  %%CITATION = HEP-PH 9907472;%%

\bibitem{vanNeerven:2000uj}
  W.~L.~van Neerven and A.~Vogt,
  %``NNLO evolution of deep-inelastic structure functions: The singlet case,''
  Nucl.\ Phys.\ B {\bf 588} (2000) 345
  [arXiv:hep-ph/0006154].
  %%CITATION = HEP-PH 0006154;%%

\bibitem{NRf90}
  Press {\it et al.}, \emph{Numerical Recipes in Fortran~90},
  Cambridge University Press, 1996.
  
\bibitem{VogtMTMParam} A.~Vogt, private communication.


\bibitem{White:2005wm}
  C.~D.~White and R.~S.~Thorne,
  %``Comparison of NNLO DIS scheme splitting functions with results from exact
  %gluon kinematics at small x,''
  Eur.\ Phys.\ J.\ C {\bf 45} (2006) 179
  [arXiv:hep-ph/0507244].
  %%CITATION = HEP-PH 0507244;%%

%\cite{Bethke:2006ac}
\bibitem{Bethke:2006ac}
  S.~Bethke,
  %``Experimental tests of asymptotic freedom,''
  Prog.\ Part.\ Nucl.\ Phys.\  {\bf 58}, 351 (2007)
  [arXiv:hep-ex/0606035].
  %%CITATION = PPNPD,58,351;%%

%\cite{de Florian:2007hc}
\bibitem{de Florian:2007hc}
  D.~de Florian, R.~Sassot and M.~Stratmann,
  %``Global Analysis of Fragmentation Functions for Protons and Charged
  %Hadrons,''
  Phys.\ Rev.\  D {\bf 76}, 074033 (2007)
  [arXiv:0707.1506 [hep-ph]].
  %%CITATION = PHRVA,D76,074033;%%


\bibitem{Corcella:2005us}
  G.~Corcella and L.~Magnea,
  %``Soft-gluon resummation effects on parton distributions,''
  Phys.\ Rev.\  D {\bf 72} (2005) 074017
  [arXiv:hep-ph/0506278].

\bibitem{Martin:2007bv}
  A.~D.~Martin, W.~J.~Stirling, R.~S.~Thorne and G.~Watt,
  %``Update of Parton Distributions at NNLO,''
  Phys.\ Lett.\  B {\bf 652}, 292 (2007)
  [arXiv:0706.0459 [hep-ph]].

%\cite{Tung:2006tb}
\bibitem{Tung:2006tb}
  W.~K.~Tung, H.~L.~Lai, A.~Belyaev, J.~Pumplin, D.~Stump and C.~P.~Yuan,
  %``Heavy quark mass effects in deep inelastic scattering and global QCD
  %analysis,''
  JHEP {\bf 0702}, 053 (2007)
  [arXiv:hep-ph/0611254].
  %%CITATION = JHEPA,0702,053;%%

\bibitem{Martin:2004dh}
  A.~D.~Martin, R.~G.~Roberts, W.~J.~Stirling and R.~S.~Thorne,
  %``Parton distributions incorporating QED contributions,''
  Eur.\ Phys.\ J.\  C {\bf 39}, 155 (2005)
  [arXiv:hep-ph/0411040].

\bibitem{Ciafaloni:2000df}
  M.~Ciafaloni, P.~Ciafaloni and D.~Comelli,
  %``Bloch-Nordsieck violating electroweak corrections to inclusive TeV  scale
  %hard processes,''
  Phys.\ Rev.\ Lett.\  {\bf 84}, 4810 (2000)
  [arXiv:hep-ph/0001142].


\bibitem{Ciafaloni:2005fm}
  P.~Ciafaloni and D.~Comelli,
  %``Electroweak evolution equations,''
  JHEP {\bf 0511}, 022 (2005)
  [arXiv:hep-ph/0505047].
  %%CITATION = JHEPA,0511,022;%%


\bibitem{FortranPolyLog}
  T.~Gehrmann and E.~Remiddi,
  %``Numerical evaluation of two-dimensional harmonic polylogarithms,''
  Comput.\ Phys.\ Commun.\  {\bf 144} (2002) 200.
  %[hep-ph/0111255].
  %%CITATION = HEP-PH 0111255;%%

\bibitem{F95Explained}
  M. Metcalf and J. Reid, \emph{Fortran 90/95 Explained}, Oxford
  University Press, 1996.

\bibitem{F95WebResources} Many introductions and tutorials about
  fortran~90 may be found at
  \url{http://dmoz.org/Computers/Programming/Languages/Fortran/Tutorials/Fortran_90_and_95/}

%============= QED refs 
  
%\cite{Manohar:2016nzj}
\bibitem{Manohar:2016nzj}
A.~Manohar, P.~Nason, G.~P.~Salam and G.~Zanderighi,
%``How bright is the proton? A precise determination of the photon parton distribution function,''
Phys. Rev. Lett. \textbf{117} (2016) no.24, 242002
doi:10.1103/PhysRevLett.117.242002
[arXiv:1607.04266 [hep-ph]].

\bibitem{Manohar:2017eqh}
A.~V.~Manohar, P.~Nason, G.~P.~Salam and G.~Zanderighi,
%``The Photon Content of the Proton,''
JHEP \textbf{12} (2017), 046
doi:10.1007/JHEP12(2017)046
[arXiv:1708.01256 [hep-ph]].

%\cite{Buonocore:2020nai}
\bibitem{Buonocore:2020nai}
L.~Buonocore, P.~Nason, F.~Tramontano and G.~Zanderighi,
%``Leptons in the proton,''
JHEP \textbf{08} (2020) no.08, 019
doi:10.1007/JHEP08(2020)019
[arXiv:2005.06477 [hep-ph]].

%\cite{Buonocore:2021bsf}
\bibitem{Buonocore:2021bsf}
L.~Buonocore, P.~Nason, F.~Tramontano and G.~Zanderighi,
%``Photon and leptons induced processes at the LHC,''
JHEP \textbf{12} (2021), 073
doi:10.1007/JHEP12(2021)073
[arXiv:2109.10924 [hep-ph]].

%==============

%\cite{Roth:2004ti}
\bibitem{Roth:2004ti}
M.~Roth and S.~Weinzierl,
%``QED corrections to the evolution of parton distributions,''
Phys. Lett. B \textbf{590} (2004), 190-198
doi:10.1016/j.physletb.2004.04.009
[arXiv:hep-ph/0403200 [hep-ph]].


%\cite{Sborlini:2013jba}
\bibitem{Sborlini:2013jba}
G.~F.~R.~Sborlini, D.~de Florian and G.~Rodrigo,
%``Double collinear splitting amplitudes at next-to-leading order,''
JHEP \textbf{01} (2014), 018
doi:10.1007/JHEP01(2014)018
[arXiv:1310.6841 [hep-ph]].

\bibitem{Sborlini:2014mpa}
G.~F.~R.~Sborlini, D.~de Florian and G.~Rodrigo,
%``Triple collinear splitting functions at NLO for scattering processes with photons,''
JHEP \textbf{10} (2014), 161
doi:10.1007/JHEP10(2014)161
[arXiv:1408.4821 [hep-ph]].

\bibitem{Sborlini:2014kla}
G.~F.~R.~Sborlini, D.~de Florian and G.~Rodrigo,
%``Polarized triple-collinear splitting functions at NLO for processes with photons,''
JHEP \textbf{03} (2015), 021
doi:10.1007/JHEP03(2015)021
[arXiv:1409.6137 [hep-ph]].

%\cite{deFlorian:2015ujt}
\bibitem{deFlorian:2015ujt}
D.~de Florian, G.~F.~R.~Sborlini and G.~Rodrigo,
%``QED corrections to the Altarelli\textendash{}Parisi splitting functions,''
Eur. Phys. J. C \textbf{76} (2016) no.5, 282
doi:10.1140/epjc/s10052-016-4131-8
[arXiv:1512.00612 [hep-ph]].

%\cite{deFlorian:2016gvk}
\bibitem{deFlorian:2016gvk}
D.~de Florian, G.~F.~R.~Sborlini and G.~Rodrigo,
%``Two-loop QED corrections to the Altarelli-Parisi splitting functions,''
JHEP \textbf{10} (2016), 056
doi:10.1007/JHEP10(2016)056
[arXiv:1606.02887 [hep-ph]].
%51 citations counted in INSPIRE as of 23 Sep 2023

%\cite{ParticleDataGroup:2022pth}
\bibitem{ParticleDataGroup:2022pth}
R.~L.~Workman \textit{et al.} [Particle Data Group],
%``Review of Particle Physics,''
PTEP \textbf{2022} (2022), 083C01
doi:10.1093/ptep/ptac097

%\cite{Frixione:2023gmf}
\bibitem{Frixione:2023gmf}
S.~Frixione and G.~Stagnitto,
%``The muon parton distribution functions,''
[arXiv:2309.07516 [hep-ph]].

%\cite{Nason:1989zy}
\bibitem{Nason:1989zy}
P.~Nason, S.~Dawson and R.~K.~Ellis,
%``The One Particle Inclusive Differential Cross-Section for Heavy Quark Production in Hadronic Collisions,''
Nucl. Phys. B \textbf{327} (1989), 49-92
[erratum: Nucl. Phys. B \textbf{335} (1990), 260-260]
doi:10.1016/0550-3213(89)90286-1
%1222 citations counted in INSPIRE as of 25 Sep 2023


\bibitem{Cacciari:2015jma}
M.~Cacciari, F.~A.~Dreyer, A.~Karlberg, G.~P.~Salam and G.~Zanderighi,
%``Fully Differential Vector-Boson-Fusion Higgs Production at Next-to-Next-to-Leading Order,''
Phys. Rev. Lett. \textbf{115} (2015) no.8, 082002
[erratum: Phys. Rev. Lett. \textbf{120} (2018) no.13, 139901]
doi:10.1103/PhysRevLett.115.082002
[arXiv:1506.02660 [hep-ph]].

\bibitem{Dreyer:2016oyx}
F.~A.~Dreyer and A.~Karlberg,
%``Vector-Boson Fusion Higgs Production at Three Loops in QCD,''
Phys. Rev. Lett. \textbf{117} (2016) no.7, 072001
doi:10.1103/PhysRevLett.117.072001
[arXiv:1606.00840 [hep-ph]].

\bibitem{Dreyer:2018qbw}
F.~A.~Dreyer and A.~Karlberg,
%``Vector-Boson Fusion Higgs Pair Production at N$^3$LO,''
Phys. Rev. D \textbf{98} (2018) no.11, 114016
doi:10.1103/PhysRevD.98.114016
[arXiv:1811.07906 [hep-ph]].

\bibitem{Dreyer:2018rfu}
F.~A.~Dreyer and A.~Karlberg,
%``Fully differential Vector-Boson Fusion Higgs Pair Production at Next-to-Next-to-Leading Order,''
Phys. Rev. D \textbf{99} (2019) no.7, 074028
doi:10.1103/PhysRevD.99.074028
[arXiv:1811.07918 [hep-ph]].

\bibitem{bertonekarlberg}
  V. Bertone and A. Karlberg, to appear.
  
\bibitem{vanNeerven:1999ca}
W.~L.~van Neerven and A.~Vogt,
%``NNLO evolution of deep inelastic structure functions: The Nonsinglet case,''
Nucl. Phys. B \textbf{568} (2000), 263-286
doi:10.1016/S0550-3213(99)00668-9
[arXiv:hep-ph/9907472 [hep-ph]].

\bibitem{Vermaseren:2005qc}
J.~A.~M.~Vermaseren, A.~Vogt and S.~Moch,
%``The Third-order QCD corrections to deep-inelastic scattering by photon exchange,''
Nucl. Phys. B \textbf{724} (2005), 3-182
doi:10.1016/j.nuclphysb.2005.06.020
[arXiv:hep-ph/0504242 [hep-ph]].

\bibitem{vanNeerven:2000uj}
W.~L.~van Neerven and A.~Vogt,
%``NNLO evolution of deep inelastic structure functions: The Singlet case,''
Nucl. Phys. B \textbf{588} (2000), 345-373
doi:10.1016/S0550-3213(00)00480-6
[arXiv:hep-ph/0006154 [hep-ph]].

\bibitem{Davies:2016ruz}
J.~Davies, A.~Vogt, S.~Moch and J.~A.~M.~Vermaseren,
%``Non-singlet coefficient functions for charged-current deep-inelastic scattering to the third order in QCD,''
PoS \textbf{DIS2016} (2016), 059
doi:10.22323/1.265.0059
[arXiv:1606.08907 [hep-ph]].

\bibitem{Moch:2004xu}
S.~Moch, J.~A.~M.~Vermaseren and A.~Vogt,
%``The Longitudinal structure function at the third order,''
Phys. Lett. B \textbf{606} (2005), 123-129
doi:10.1016/j.physletb.2004.11.063
[arXiv:hep-ph/0411112 [hep-ph]].

\bibitem{Moch:2008fj}
S.~Moch, J.~A.~M.~Vermaseren and A.~Vogt,
%``Third-order QCD corrections to the charged-current structure function F(3),''
Nucl. Phys. B \textbf{813} (2009), 220-258
doi:10.1016/j.nuclphysb.2009.01.001
[arXiv:0812.4168 [hep-ph]].

%\cite{Moch:2021qrk}
\bibitem{Moch:2021qrk}
S.~Moch, B.~Ruijl, T.~Ueda, J.~A.~M.~Vermaseren and A.~Vogt,
%``Low moments of the four-loop splitting functions in QCD,''
Phys. Lett. B \textbf{825} (2022), 136853
doi:10.1016/j.physletb.2021.136853
[arXiv:2111.15561 [hep-ph]].
%32 citations counted in INSPIRE as of 05 Sep 2023

%\cite{Falcioni:2023luc}
\bibitem{Falcioni:2023luc}
G.~Falcioni, F.~Herzog, S.~Moch and A.~Vogt,
%``Four-loop splitting functions in QCD \textendash{} The quark-quark case,''
Phys. Lett. B \textbf{842} (2023), 137944
doi:10.1016/j.physletb.2023.137944
[arXiv:2302.07593 [hep-ph]].
%5 citations counted in INSPIRE as of 05 Sep 2023

%\cite{Falcioni:2023vqq}
\bibitem{Falcioni:2023vqq}
G.~Falcioni, F.~Herzog, S.~Moch and A.~Vogt,
%``Four-loop splitting functions in QCD -- The gluon-to-quark case,''
[arXiv:2307.04158 [hep-ph]].
%1 citations counted in INSPIRE as of 05 Sep 2023

%\cite{Gehrmann:2023cqm}
\bibitem{Gehrmann:2023cqm}
T.~Gehrmann, A.~von Manteuffel, V.~Sotnikov and T.~Z.~Yang,
%``Complete $N_f^2$ contributions to four-loop pure-singlet splitting functions,''
[arXiv:2308.07958 [hep-ph]].
%0 citations counted in INSPIRE as of 05 Sep 2023

%\cite{Blumlein:2022gpp}
\bibitem{Blumlein:2022gpp}
J.~Bl\"umlein, P.~Marquard, C.~Schneider and K.~Sch\"onwald,
%``The massless three-loop Wilson coefficients for the deep-inelastic structure functions F$_{2}$, F$_{L}$, xF$_{3}$ and g$_{1}$,''
JHEP \textbf{11} (2022), 156
doi:10.1007/JHEP11(2022)156
[arXiv:2208.14325 [hep-ph]].
%15 citations counted in INSPIRE as of 06 Sep 2023

%\cite{Falcioni:2023tzp}
\bibitem{Falcioni:2023tzp}
G.~Falcioni, F.~Herzog, S.~Moch, J.~Vermaseren and A.~Vogt,
%``The double fermionic contribution to the four-loop quark-to-gluon splitting function,''
Phys. Lett. B \textbf{848} (2024), 138351
doi:10.1016/j.physletb.2023.138351
[arXiv:2310.01245 [hep-ph]].
%2 citations counted in INSPIRE as of 04 Dec 2023

%\cite{Moch:2023tdj}
\bibitem{Moch:2023tdj}
S.~Moch, B.~Ruijl, T.~Ueda, J.~Vermaseren and A.~Vogt,
%``Additional moments and x-space approximations of four-loop splitting functions in QCD,''
[arXiv:2310.05744 [hep-ph]].
%3 citations counted in INSPIRE as of 04 Dec 2023

%\cite{Gehrmann:2023iah}
\bibitem{Gehrmann:2023iah}
T.~Gehrmann, A.~von Manteuffel, V.~Sotnikov and T.~Z.~Yang,
%``The $N_f \,C_F^3$ contribution to the non-singlet splitting function at four-loop order,''
[arXiv:2310.12240 [hep-ph]].
%0 citations counted in INSPIRE as of 04 Dec 2023

%\cite{Salam:2008qg}
\bibitem{Salam:2008qg}
G.~P.~Salam and J.~Rojo,
%``A Higher Order Perturbative Parton Evolution Toolkit (HOPPET),''
Comput. Phys. Commun. \textbf{180} (2009), 120-156
doi:10.1016/j.cpc.2008.08.010
[arXiv:0804.3755 [hep-ph]].
%271 citations counted in INSPIRE as of 04 Dec 2023

\end{thebibliography}


\end{document}

