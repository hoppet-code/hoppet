\section{Perturbative evolution in QCD}
\label{sec:pqcd}

\ifreleasenote%
  \textbf{Note:} this section is largely a repetition of Section~2 of
  Ref.~\cite{Salam:2008qg}, and is included again here to help provide
  context for the discussion that follows in subsequent
  sections.\medskip
%
\fi

First of all we set up the notation and
conventions that are used throughout \hoppet. The DGLAP
equation for a full flavour set of parton distributions, $q$, reads
\begin{equation}
  \label{eq:dglap-ns}
  \frac{\partial q(x,Q^2)}{\partial \ln Q^2} = 
  \int_x^1 \frac{dz}{z}
  P\left(z,\aq \right) q\lp \frac{x}{z},Q^2\rp \equiv 
  P\left(x,\aq\right) \otimes q\lp x,Q^2\rp,
\end{equation}
where $P$ is a matrix in flavour space and is expressed as an
expansion in powers of the strong coupling $\alpha_s$
\begin{equation}
  \label{eq:dpdf}
  P\left( z,\aq \right)
  =
  \sum_{n=1}   \lp \frac{\aq}{2\pi} \rp^n P^{(n)}(z)
  %
  %\frac{\aq}{2\pi} P^{(1)}(z)
  % +
  % +
  % \lp \frac{\aq}{2\pi} \rp^3 P^{(3)}(z) 
  % +
  % \lp \frac{\aq}{2\pi} \rp^4 P^{(4)}(z) \ ,
\end{equation}
%
The splitting functions in eq.~(\ref{eq:dglap-ns}) are known exactly
up to NNLO ($n\le 3$) in the unpolarised case
\cite{Furmanski:1980cm,Curci:1980uw,NNLO-NS,NNLO-singlet}, and
approximately at
N$^3$LO ($n\le 4$)~\cite{Gracey:1994nn,Davies:2016jie,Moch:2017uml,Gehrmann:2023cqm,Falcioni:2023tzp,Gehrmann:2023iah,McGowan:2022nag,NNPDF:2024nan,Moch:2021qrk,Falcioni:2023luc,Falcioni:2023vqq,Moch:2023tdj,Falcioni:2024xyt,Falcioni:2024qpd,Falcioni:2025hfz};
and up to NNLO
\cite{Mertig:1995ny,Vogelsang:1996im,Moch:2014sna,Moch:2015usa,Blumlein:2021enk,Blumlein:2021ryt}
in the polarised case.  The generalisation to the singlet case is
straightforward, as it is
%the generalisation of eq.~(\ref{eq:dglap-ns}) 
to the case of time-like evolution,\footnote{
The general structure of the relation between space-like
and time-like evolution and splitting functions
 has been investigated in
 \cite{Furmanski:1980cm,Curci:1980uw,Stratmann:1996hn,Dokshitzer:2005bf,Mitov:2006ic,Basso:2006nk,Dokshitzer:2006nm,Beccaria:2007bb}.
 %
 See references to those articles for more recent updates.
}
relevant for example for fragmentation function analysis,
where NNLO results
are also available \cite{Mitov:2006ic,Moch:2007tx,Almasy:2011eq}.

In terms of the flavour structure of Eq.~(\ref{eq:dglap-ns}), there
are different possible flavour bases for representing both the parton
distributions $q$ and the splitting functions.
%
Using $q_i$ to denote the $i^\text{th}$ quark, $i=1,\ldots,6$, $\bar
q_i$ for the corresponding anti-quark, and $g$ for the gluon, it is
common to use a basis expressed in terms of flavour and anti-flavour
combinations,
\begin{subequations}
  \label{eq:flavour-basis}
  \begin{align}
    q_{i}^\pm            &= q_i \pm \bar q_i\,, \\
    \intertext{and from those, non-singlet (NS), singlet ($\Sigma$) and valence (V) combinations~\cite{vanNeerven:1999ca,vanNeerven:2000uj},}
    q_{\NS,ik}^\pm &= q_i^\pm - q_k^\pm\,, \\
    q^V_\NS        &= \sum_{i=1}^{n_f} q_i^-\,,\\
    \Sigma               &= \sum_{i=1}^{n_f} q_i^+\,,
  \end{align}
\end{subequations}
with the evolution then taking the form
\begin{subequations}\label{eq:P-flavour-basis}
  \begin{alignat}{2}
    &\partial_{\ln Q^2} \, q_{\NS,ik}^\pm &&= P_\NS^\pm \otimes q_{\NS,ik}^\pm\,, \\
    &\partial_{\ln Q^2} \, q_{\NS}^V      &&= P_\NS^V   \otimes q_{\NS}^V\,, \\
    &\partial_{\ln Q^2} \, \Sigma         &&= P_{qq}    \otimes \Sigma + P_{qg} \otimes g\,\\
    &\partial_{\ln Q^2} \, g              &&= P_{gq}    \otimes \Sigma + P_{gg} \otimes g\,.
  \end{alignat}
\end{subequations}
The QCD coupling also satisfies a renormalisation group equation,
\begin{equation}
  \label{eq:as-ev}
  \frac{d\as}{d\ln Q^2} = \beta\lp \aq\rp = -\as (\beta_0\as +
  \beta_1\as^2 + 
  \beta_2\as^3 + 
  \beta_3\as^4) \,.
\end{equation}
Generally, our convention in \hoppet will be that perturbative quantities 
are defined to be coefficients of powers of $\as/2\pi$, but
note the different convention specifically in the case of the
$\beta$-function in Eq.~(\ref{eq:as-ev}).
%
This is for historical reasons and has been maintained because of the
significant number of external codes that rely on that convention.

% As with the splitting functions, most perturbative quantities in
% \hoppet are defined to be coefficients of powers of $\as/2\pi$.
% %
% However there are some places where a different convention is used,
% either for historical reasons or because external code uses a
% different convention.
% %
% In particular the $\beta$-function coefficients of the running
% coupling equation,
% \begin{equation}
%   \label{eq:as-ev}
%   \frac{d\as}{d\ln Q^2} = \beta\lp \aq\rp = -\as (\beta_0\as +
%   \beta_1\as^2 + 
%   \beta_2\as^3 + 
%   \beta_3\as^4) \,,
% \end{equation}
% are defined internally in \hoppet as multiplying powers of $\alpha_s$
% directly.

The evolution of the strong coupling and the parton distributions can
be performed in both the fixed flavour-number scheme (FFNS) and the 
variable flavour-number scheme (VFNS). In the VFNS case we 
need the matching conditions between the effective
theories with $n_f$ and $n_{f}+1$ light flavours for both the strong 
coupling $\aq$ and the parton distributions at the heavy quark
mass threshold $m_h^2$.

These matching conditions for the parton distributions
receive non-trivial contributions at higher orders.
%
To understand the flavour structure of the matching conditions, it is
useful to introduce $q_{i\le n_\ell}^{(n_f)}$ to refer to the gluon
and quark flavours with $i\le n_\ell$ in a effective theory with $n_f$
flavours.
%
Then in going from $n_\ell \to n_h = n_\ell+1$ flavours, the matching
condition takes the form
\begin{align}
  \label{eq:lp-nf1}
  q_{i\le n_\ell}^{(n_h)} = q_{i\le n_\ell}^{(n_\ell)} + A_{i\le n_\ell} \otimes q_{i\le n_\ell}^{(n_h)}
\end{align}
for the light flavours and the gluon, where the light-flavour part of
the matching matrix, $A_{i\le n_\ell}$, has the
same flavour entries, and acts in the same way as an $n_\ell$-flavour
splitting matrix, cf.\ Eq.~(\ref{eq:P-flavour-basis}).
%
Additionally the $\pm$ combinations for heavy-flavour $h$ in the
$n_h$-flavour effective theory, $q_h^{\pm(n_h)}$, are given by
\begin{subequations}
  \label{eq:hp-nf1}
  \begin{align}
    q_h^{+(n_h)} &= A_{h\Sigma} \otimes \Sigma^{(n_\ell)} + A_{hg} \otimes g^{(n_\ell)} \\
    q_h^{-(n_h)} &= A_{hV     } \otimes q_\NS^{V(n_\ell)}
  \end{align}
\end{subequations}
As with the splitting functions, the matching matrices are calculated
as an expansion in powers of the strong coupling.
%
In the $\MSbar$ (factorisation) scheme, carrying out the matching at a
scale equal to the heavy-quark mass, the series starts at order
$\alpha_s^2$~\cite{NNLO-MTM}, which in terms of the logarithmic accuracy is equivalent
to NNLO in the evolution.
%
The matching matrices are currently known up to
\ntlo~\cite{Bierenbaum:2009mv,Ablinger:2010ty,Kawamura:2012cr,Blumlein:2012vq,ABLINGER2014263,Ablinger:2014nga,Ablinger:2014vwa,Behring:2014eya,Ablinger:2019etw,Behring:2021asx,Fael:2022miw,Ablinger:2023ahe,Ablinger:2024xtt,BlumleinCode}.
%
The exact functional form depends on the choice of scheme for the
mass, e.g.\ pole v.\ $\MSbar$ mass.\footnote{In a general
  factorisation scheme, and when matching at scales other than the
  heavy-quark mass, the matching terms start at order $\as$.}
%
Note that at lower orders some of the flavour structures are trivially
equal to others, or are zero (e.g.\ $A_{hV}$ is zero at NNLO).

As is evident from Eqs.~(\ref{eq:lp-nf1}) and (\ref{eq:hp-nf1}), the matching
conditions lead to small discontinuities of the PDFs between the
$n_\ell$ and $n_h$-flavour effective theories.
%
Similarly, from NNLO onwards, the heavy-quark PDFs start from a
non-zero value at threshold, which sometimes can even be negative.
%
Physical predictions intended to be used at scales close to a heavy
quark mass and that take into account full quark-mass effects
generally include terms that compensate for these effects as one
crosses the mass threshold.
%
Most DGLAP evolution codes take the approximation that different heavy
quarks are well separated in mass.
%
However if one relaxes this assumption, then there are additional
terms that correlate nearby thresholds of different heavy
quarks~\cite{Ablinger:2025nnq}.
%
These are not yet included in \hoppet.


% its
% evolution in $Q^2$, which are cancelled by similar matching terms in
% the coefficient functions in massive VFN schemes, resulting in continuous physical
% observables. In particular, the heavy-quark PDFs start from a non-zero
% value at threshold at NNLO, which sometimes can even be negative.
% 
% for light quarks $q_{l,i}$ of flavour $i$ (quarks that are considered
% massless below the heavy quark mass threshold $m_h^2$) the matching
% between their values in the $n_f$ and $n_f+1$ effective theories
% reads:\footnote{Note that the literature focuses on the
%   $q_{l,i}+q_{l,-i}$ combination. We thank Johannes Bl\"umlein for
%   having provided the $A^{\rm NS,-}_{qq,h}(x)$ code needed for the
%   $q_{l,i}-q_{l,-i}$ combination.}:
% %\begin{equation}
% %\label{eq:lp-nf1}
% %  q_{l,i}^{\,(n_f+1)}(x,m_h^2) \: = \:  q_{l,i}^{\,(\nf}(x,m_h^2) +
% %\lp \frac{\alpha_s(m_h^2)}{2\pi} \rp^2
% %   A^{\rm ns,(2)}_{qq,h}(x) \otimes
% %  q_{l,i}^{\, (\nf}(x,m_h^2) \ ,
% %\end{equation}
% \begin{align}
% \label{eq:lp-nf1}
%   q_{l,i}^{\,(n_f+1)}(x,m_h^2) + q_{l,-i}^{\,(n_f+1)}(x,m_h^2)  & =   q_{l,i}^{\,(\nf}(x,m_h^2) + q_{l,-i}^{\,(\nf}(x,m_h^2) \notag \\ &+
%    A^{\rm NS,+}_{qq,h}(x) \otimes \left(
%    q_{l,i}^{\, (\nf}(x,m_h^2) + q_{l,-i}^{\, (\nf}(x,m_h^2)\right) \notag\\
%    & + \frac{1}{n_f} \Big\{A^{\rm PS}_{qq,h}(x) \otimes \Sigma^{\, (\nf}(x,m_h^2) \notag\\
%    & + A^{\rm S}_{qg,h}(x) \otimes g^{\, (\nf}(x,m_h^2)\Big\} \ , \notag \\
%   q_{l,i}^{\,(n_f+1)}(x,m_h^2) - q_{l,-i}^{\,(n_f+1)}(x,m_h^2)  & =   q_{l,i}^{\,(\nf}(x,m_h^2) - q_{l,-i}^{\,(\nf}(x,m_h^2) \notag \\ &+
%    A^{\rm NS,-}_{qq,h}(x) \otimes \left(
%    q_{l,i}^{\, (\nf}(x,m_h^2) - q_{l,-i}^{\, (\nf}(x,m_h^2)\right) \ ,
% \end{align}
% where $i = 1,\ldots n_f$, while for the gluon distribution, the heavy
% quark PDF $q_h$, and the singlet PDF $\Sigma(x,Q^2)$ (defined in
% Table~\ref{eq:diag_split}) one has :
% \begin{align}
% \label{eq:hp-nf1}
%   g^{(n_f+1)}(x,m_h^2)  &=
%     g^{\, (\nf}(x,m_h^2) +
%     A_{\rm gq,h}^{\rm S}(x) \otimes \Sigma^{(\nf}(x,m_h^2) +
%     A_{\rm gg,h}^{\rm S}(x) \otimes g^{(\nf}(x,m_h^2) \ ,
%   \nn \\[0.3cm]
%   (q_h+\bar{q}_{h})^{(n_f+1)}(x,m_h^2)  &=
%   A_{\rm hq}^{\rm S}(x)\otimes \Sigma^{(\nf}(x,m_h^2) 
%   + A_{\rm hg}^{\rm S}(x)\otimes g^{(\nf}(x,m_h^2)\ ,  \nn \\[0.3cm]
%   \Sigma^{(n_f+1)}(x,m_h^2)  &= \Sigma^{\, (\nf}(x,m_h^2) + \left[ A^{\rm NS,+}_{qq,h}(x) + A^{\rm PS}_{qq,h}(x) + A_{\rm hq}^{\rm S}(x)\right] \otimes \Sigma^{(\nf}(x,m_h^2) \nn \\
%   & + \left[ A^{\rm S}_{qg,h}(x) + A_{\rm hg}^{\rm S}(x) \right] \otimes g^{(\nf}(x,m_h^2)
% \end{align}
% with $q_h=\bar{q}_h$. Up to N$^3$LO the matching coefficients have the
% following expansions in $\alpha_s$
% \begin{align}
%   A^{\rm NS,\pm}_{qq,h}(x) & = \lp \frac{\alpha_s(m_h^2)}{2\pi} \rp^2
%   A^{\rm NS,\pm,(2)}_{qq,h}(x) + \lp \frac{\alpha_s(m_h^2)}{2\pi} \rp^3
%   A^{\rm NS,\pm,(3)}_{qq,h}(x) \ , \nn \\
%   A^{\rm S}_{gk,h}(x) & = \lp \frac{\alpha_s(m_h^2)}{2\pi} \rp^2
%   A^{\rm S,(2)}_{gk,h}(x) + \lp \frac{\alpha_s(m_h^2)}{2\pi} \rp^3
%   A^{\rm S,(3)}_{gk,h}(x), \quad k=q,g \ , \nn \\
%   %A^{\rm S}_{gg,h}(x) & = \lp \frac{\alpha_s(m_h^2)}{2\pi} \rp^2
%   %A^{\rm S,(2)}_{gg,h}(x) + \lp \frac{\alpha_s(m_h^2)}{2\pi} \rp^3
%   %A^{\rm S,(3)}_{gg,h}(x) \ , \nn \\
%   A^{\rm S}_{hk}(x) & = \lp \frac{\alpha_s(m_h^2)}{2\pi} \rp^2
%   A^{\rm S,(2)}_{hk}(x) + \lp \frac{\alpha_s(m_h^2)}{2\pi} \rp^3
%   A^{\rm S,(3)}_{hk}(x), \quad k=q,g \ , \nn \\
%   %A^{\rm S}_{hg}(x) & = \lp \frac{\alpha_s(m_h^2)}{2\pi} \rp^2
%   %A^{\rm S,(2)}_{hg}(x) + \lp \frac{\alpha_s(m_h^2)}{2\pi} \rp^3
%   %A^{\rm S,(3)}_{hg}(x) \ , \nn \\
%   A^{\rm PS}_{qq,h}(x) & = \lp \frac{\alpha_s(m_h^2)}{2\pi} \rp^3
%   A^{\rm PS,(3)}_{qq,h}(x) \, , \nn \\
%   A^{\rm S}_{qg,h}(x) & = \lp \frac{\alpha_s(m_h^2)}{2\pi} \rp^3
%   A^{\rm S,(3)}_{qg,h}(x)\,.
% \end{align}
% At $\mathcal{O}(\alpha_S^2)$ we have that $A^{\rm NS,+}_{qq,h}(x) =
% A^{\rm NS,-}_{qq,h}(x)$ whereas they start to differ at
% $\mathcal{O}(\alpha_S^3)$. The NNLO matching coefficients were
% computed in \cite{NNLO-MTM}\footnote{We thank the authors for the
% code corresponding to the calculation.} and the N$^3$LO matching
% coefficients
% in~\cite{Bierenbaum:2009mv,Ablinger:2010ty,Kawamura:2012cr,Blumlein:2012vq,ABLINGER2014263,Ablinger:2014nga,Ablinger:2014vwa,Behring:2014eya,Ablinger:2019etw,Behring:2021asx,Fael:2022miw,Ablinger:2023ahe,Ablinger:2024xtt,BlumleinCode}.\footnote{We
% thank Johannes Bl\"umlein for sharing a pre-release version the code from
% Ref.~\cite{BlumleinCode} with us, which also contains code associated
% with Refs.~\cite{Ablinger:2024xtt,Fael:2022miw}.
% %
% }
% %
% Notice that the above
% conditions will lead to small discontinuities of the PDFs in its
% evolution in $Q^2$, which are cancelled by similar matching terms in
% the coefficient functions in massive VFN schemes, resulting in continuous physical
% observables. In particular, the heavy-quark PDFs start from a non-zero
% value at threshold at NNLO, which sometimes can even be negative.

The corresponding N$^3$LO relation for the matching of the $\MSbar$
coupling constant at the heavy quark threshold $m^2_h$ is given by 
\begin{equation}
\label{eq:as-nf1}
  \as^{\, (n_f+1)}(m_h^2) \: = \:
  \as^{\, (\nf} (m_h^2) +   C_2 \lp \frac{\as^{\, (\nf} (m_h^2)}{2\pi} \rp^3+   C_3 \lp \frac{\as^{\, (\nf} (m_h^2)}{2\pi} \rp^4
   \:\: ,
\end{equation}
where the matching coefficients $C_2$ and $C_3$ were computed in
\cite{Chetyrkin:1997sg,Chetyrkin:1997un}.
%
The value and the form of the matching coefficients in
eqs.~(\ref{eq:lp-nf1},\ref{eq:hp-nf1}) depend on the scheme used for
the quark masses; by default in \hoppet quark masses are taken to be
pole masses, though the option exists for the user to supply and have
thresholds crossed at $\MSbar$ masses, but only up to NNLO. We note
that in the current implementation in \hoppet, the matching can only be
performed at the matching point that corresponds to the heavy-quark masses
themselves.
%
As for the PDFs, when two heavy-quark thresholds are considered close,
there are additional terms from the simultaneous decoupling of the two
heavy quarks~\cite{Grozin:2011nk}, which are not yet included in
\hoppet. 

Both evolution and threshold matching preserve the momentum sum rule
\begin{equation}
  \int_0^1 dx~x \lp \Sigma(x,Q^2)+g(x,Q^2)\rp =1 \,,
\end{equation}
and valence sum rules
\begin{equation}
  \int_0^1 dx\, \left[q(x,Q^2)-{\bar q}(x,Q^2) \right] = \left\{ 
    \begin{array}{ll}
      1, & \text{for } q = d \text{ (in proton)}\\
      2, & \text{for } q = u \text{ (in proton)}\\
      0, & \text{other flavours}
    \end{array}
    \right.
\end{equation}
as long as they hold at the initial scale (occasionally not the case,
\eg in modified LO sets for Monte Carlo
generators~\cite{Sherstnev:2008dm}).

% The default basis for the PDFs, called the \ttt{human} 
% representation in \hoppet, is such that 
%  the entries in an array
% \ttt{pdf(-6:6)} of PDFs correspond to:
% \bea 
% \bar t={-6} \ ,  \bar b={-5} \ ,  \bar c={-4}
% \ , \nn   \bar s&=&{-3} \ , \nn  \bar u={-2} \ , \nn
%  \bar d={-1} \ , \\  g&=&{0} \ , \\ \nn   d={1} \ , \nn  u={2} 
% \ , \nn  
% s={3} \ , \nn   c&=&{4} \ , \nn b={5} \ , \nn  t={6} \ . \nn 
% \eea
% %  This representation is the
% % same as that used in the \ttt{LHAPDF} library \cite{LHAPDF}. 
% %
% However, this representation leads
% to a complicated form of the evolution equations.
% The splitting matrix can be simplified considerably (made diagonal
% except for a $2\times2$ singlet block) by switching to a different
% flavour representation, which is named
% the \ttt{evln} representation, for the PDF set, as explained in detail in
% \cite{vanNeerven:1999ca,vanNeerven:2000uj}. This representation
% is described in Table \ref{eq:diag_split}.
% 
% In the {\tt evln} basis, 
% the gluon evolves coupled to the singlet  PDF $\Sigma$,
% and all non-singlet PDFs evolve independently.
% Notice that the representations of the PDFs
% are preserved under linear operations, so in particular
% they are preserved under DGLAP evolution.
% The conversion from the \ttt{human} to the \ttt{evln}
% representations of PDFs requires that the number of
% active quark flavours $n_f$ be specified by the user, as described in
% \ifreleasenote
% Section~5.1.2 of Ref.~\cite{Salam:2008qg}.
% \else
% Section~\ref{sec:evln-rep}.
% \fi
% 
% \begin{table}
% \begin{center}
% \begin{tabular}{|r | c | l |}
% \hline
%      i & \mbox{name} & $q_i$ \\ \hline
%      $ -6\ldots-(n_f+1)$ & $q_i$ & $q_i$\\
%      $-n_f\ldots -2$ & $q_{\mathrm{NS},i}^{-}$ & 
% $(q_i -  {\bar q}_i) - (q_1 - {\bar q}_1)$\\
%       -1           & $q_{\mathrm{NS}}^{V}$ & 
% $\sum_{j=1}^{n_f} (q_j -  {\bar q}_j)$\\
%        0           & g & \textrm{gluon} \\
%        1           & $\Sigma$ & $\sum_{j=1}^{n_f} (q_j +  {\bar q}_j)$\\
%      $2\ldots n_f$ & $q_{\mathrm{NS},i}^{+}$ &
% $ (q_i +  {\bar q}_i) - (q_1 + {\bar q}_1)$\\
%       $(n_f+1)\ldots6$ & $q_i$ & $q_i$ \\
% \hline
% \end{tabular}
% \caption{}{\label{eq:diag_split} The evolution representation 
% (called \ttt{evln} in \hoppet)
% of PDFs with $n_f$ active quark flavours
% in terms of the \ttt{human} representation.}  
% \end{center}
% \end{table}
 
In \hoppet, unpolarised DGLAP evolution is available up to N$^3$LO
in the $\MSbar$ scheme, while for the DIS scheme
only evolution up to NLO is available, but without the NLO heavy-quark
threshold matching conditions. For polarised evolution up to NLO only
the $\MSbar$ scheme is available. The variable \ttt{factscheme}
takes different values for each factorisation scheme:
\begin{center}
  \begin{tabular}{|c|l|}\hline
    \ttt{factscheme} & Evolution\\[2pt]\hline
    1 & unpolarised $\MSbar$ scheme\\[2pt]\hline
    2 & unpolarised DIS scheme\\[2pt]\hline
    3 & polarised $\MSbar$ scheme\\\hline
  \end{tabular}
\end{center}
Note that mass thresholds are currently
missing in the DIS scheme.

The extension to QED is conceptually straightforward.
%
Further discussion of that is given in
Section~\ref{sec:qed-evolution}.


%%% Local Variables:
%%% TeX-master: "HOPPET-doc.tex"
%%% End:
