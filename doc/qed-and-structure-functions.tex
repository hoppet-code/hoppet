% ======================================================================
\section{Evolution including QED contributions}
\label{sec:qed-evolution}

The combined QED+QCD evolution, as implemented in \hoppet since
version 2.0.0 (and earlier in a dedicated \ttt{qed} branch), was first
described in
Refs.~\cite{Manohar:2016nzj,Manohar:2017eqh,Buonocore:2020nai,Buonocore:2021bsf}.
%
The determination of which contributions to include follows a
consistent approach based on the so-called ``phenomenological'' counting
scheme.
%
Within this scheme, one considers the QED coupling $\alpha$ to be of order $\as^2$, and takes the
photon (lepton) PDF to be of order $\alpha L$ ($\alpha^2 L^2$), where
$L$ is the logarithm of the
ratio of the factorisation scale to a typical hadronic scale and is considered to be of order $L \sim 1/\as$.
In contrast, quark and gluon
PDFs are considered to be of order $(\as L)^n={\cal O}(1)$.\footnote{The
  above counting is to be contrasted with a ``democratic'' scheme, in
  which one considers $\alpha \sim \as \sim 1/L$ and where the aim would be to maintain
  the same loop order across all couplings and splitting functions,
  regardless of the relative numbers of QCD and QED couplings that
  they involved.}
%
From this point of view, NNLO (3-loop) QCD evolution provides control of
terms of order up to $\as^{n+2} L^n \sim \as^2$.
%
To achieve a corresponding accuracy when including QED contributions,
\hoppet has been extended to account for
\begin{enumerate}
\item \label{item:qed1} 1-loop QED splitting functions~\cite{Roth:2004ti}, which first
  contribute at order $\alpha L \sim \as$, i.e.\ count as NLO QCD
  corrections;
  
\item \label{item:qed2} 1-loop QED running coupling, including lepton and quark
  thresholds, which first contributes at order $\alpha^2 L^2 \sim
  \as^2$, i.e. like NNLO QCD; 
  
\item \label{item:qed3} 2-loop mixed QCD-QED splitting
  functions~\cite{deFlorian:2015ujt},
  %\footnote{One-loop triple collinear splitting functions with photons have been computed in refs~\cite{Sborlini:2014mpa,Sborlini:2014kla}};,
  which first contribute at order
  $\alpha \as L \sim \as^2$, i.e.\ count as NNLO QCD corrections;

\item \label{item:qed4} optionally, the 2-loop pure QED $P_{\ell q}$ splitting
  function~\cite{deFlorian:2016gvk}, which brings absolute accuracy
  $\alpha^2 L\sim \as^3$ to the lepton distribution (which starts at
  $\alpha^2 L^2 \sim \as^2$).\footnote{The implementation of the 2-loop $P_{\ell q}$ splitting
  function in the hoppet code (\ttt{Plq\_02} in the code) was carried out by Luca Buonocore.}
  %
  In an absolute counting of accuracy, this is not needed.
  % 
  However, if one wants lepton distributions to have the same relative
  NLO accuracy as the photon distribution, it should be
  included.
\end{enumerate}
%
The code could be extended systematically to aim at a higher
accuracy. For instance, if one wished to reach N$^3$LO accuracy in the
phenomenological counting, one would need to include 3-loop mixed
QCD-QED splitting functions at order $\alpha \as^2$, which contribute
at order $\alpha \as^2 L \sim \as^3$ but are currently not
available, and the well-known 2-loop QED running coupling at order $\alpha \alpha_s$, which contributes at order $\alpha^2 \as L^2 \sim \as^3$.
%\commentpn{I would put the comma differently (as before, Gavin decides):
%   at order $\alpha \as^2$, which contribute
%at order $\alpha \as^2 L \sim \as^3$ but are currently not
%available, and the well-known 2-loop QED running coupling at order $\alpha \alpha_s$, which contributes at order $\alpha^2 \as L^2 \sim \as^3$.}
% 
%\comment{Discuss what is needed to go effectively to N3LO? $\alpha^2$
%  splitting exists, $\alpha\as^2$ does not?}
%


\subsection{Implementation of the QED extension}

\myparagraph{QED coupling}

A first ingredient is the setup of the QED coupling object, defined in
\ttt{module qed\_coupling}:
\begin{lstlisting}
  type qed_coupling 
     real(dp) :: m_light_quarks
     real(dp) :: mc, mb, mt
     integer  :: n_thresholds
     integer  :: nflav(3, 0:n_thresholds) ! first index: 1 = nleptons, 2=ndown, 3=nup
     real(dp) :: thresholds(0:n_thresholds+1)
     real(dp) :: b0_values(n_thresholds)
     real(dp) :: alpha_values(n_thresholds)
  end type qed_coupling
\end{lstlisting}
%
This is initialized through a call to 
\begin{lstlisting}
  subroutine InitQEDCoupling(coupling, m_light_quarks, &
   &                           m_heavy_quarks(4:6) [,value_at_scale_0])
    type(qed_coupling), intent(out) :: coupling
    real(dp),           intent(in)  :: m_light_quarks, m_heavy_quarks(4:6)
    real(dp), optional, intent(in)  :: value_at_scale_0 ! defaults to alpha_qed_scale_0
    [...]
  end subroutine InitQEDCoupling
\end{lstlisting}
%
It initialises the parameters relevant to the QED coupling and its
running.  The electromagnetic coupling at scale zero is set by default
to its PDG Thomson value~\cite{ParticleDataGroup:2022pth} value, unless the optional
argument \ttt{value\_at\_scale\_0} is provided, in which case the
latter is taken.
% as the value of the QED coupling at zero momentum.
%\commentpn{prefer:  is set by default
%to its PDG~\cite{ParticleDataGroup:2022pth} Thomson value, unless the optional
%argument \ttt{value\_at\_scale\_0} is provided, in which case the
%latter is used.}

The running is performed at leading order level using seven
thresholds: a common effective mass for the three light quarks
(\ttt{m\_light\_quarks}), the three lepton masses (hard-coded to their
PDG values~\cite{ParticleDataGroup:2022pth} in the
\ttt{src/qed\_coupling.f90} file), and the three masses of the heavy
quarks (\ttt{m\_heavy\_quarks(4:6)}).
%
The common value of the light quark masses is used to mimic the
physical evolution in the region $0.1\GeV \lesssim \mu \lesssim
1\GeV$, which involves hadronic states.
%
Using a value of $0.109\GeV$ generates QED coupling values at
the masses of the $\tau$ lepton ($1/133.458$) and $Z$-boson ($1/127.952$) that
agree with the PDG $\MSbar$  ones ($1/(133.471\pm0.007)$ and $1/(127.951\pm0.009)$)
with a relative $\sim 10^{-4}$ accuracy.
%\gps{We don't really have a
%  QED renormalisation scheme defined and $1/137$ that we use for $\alpha(0)$
%  is not, I think, an $\MSbar$ value; hence the rephrasing putting
%  MSbar in front of PDG rather than in front of our coupling.}
%\commentpn{Actually the Thomson value IS equal the MSbar value below all thresholds.}
% 
%Lepton masses are hard-coded to their PDG
%values~\cite{ParticleDataGroup:2022pth} in the
%\ttt{src/qed\_coupling.f90} file.
%
%The heavy-quark masses \ttt{quark\_masses(4:6)} are instead required as
%non-optional arguments.

The quark and lepton masses are used to set all thresholds where the
fermion content changes. This information is then used to set
the array \ttt{nflav(3,0:n\_thresholds)}, where \ttt{nflav(1:3,i)}
contains an integer array with the number of leptons, down, and up
quarks at a given $Q^2$ such that \ttt{threshold(i-1)}$<Q^2<$
\ttt{threshold(i)}.
%
The \ttt{threshold(1:7)} entries are active thresholds, while
\ttt{threshold(0)} is set to zero and \ttt{threshold(8)} to an
arbitrary large number (currently $10^{200}$).
%
The integer array
\ttt{nflav(1:3,0:n\_thresholds)} is then used to compute the
$\beta_{0,\rm QED}$ function \ttt{b0\_values(1:n\_thresholds)} at the
seven threshold values and this is finally used to compute the value
of the QED coupling \ttt{alpha\_values(1:n\_thresholds)} at the
threshold values.
%
The function \ttt{Delete(qed\_coupling)} is also provided for
consistency, although in this case it does nothing.
%
After this initialization, the function \ttt{Value(qed\_coupling,mu)} returns the QED coupling at
scale $\mu$.

\myparagraph{QED splitting matrices}

The QED splitting matrices are stored in the object

\begin{lstlisting}
  type qed_split_mat
     type(qed_split_mat_lo)   :: lo
     type(qed_split_mat_nlo)  :: nlo
     type(qed_split_mat_nnlo) :: nnlo
  end type qed_split_mat
\end{lstlisting}
defined in \ttt{qed\_objects.f90}. 
This contains the LO, NLO and NNLO splitting matrices 
\begin{lstlisting}
  ! a leading-order splitting matrix (multiplies alpha/2pi)
  type qed_split_mat_lo
     type(grid_conv) :: Pqq_01, Pqy_01, Pyq_01, Pyy_01
     integer         :: nu, nd, nl, nf
  end type qed_split_mat_lo

  ! a NLO splitting matrix (multiplies (alpha alpha_s)/(2pi)^2)
  type qed_split_mat_nlo
     type(grid_conv) :: Pqy_11, Pyy_11, Pgy_11
     type(grid_conv) :: Pqg_11, Pyg_11, Pgg_11
     type(grid_conv) :: PqqV_11, PqqbarV_11, Pgq_11, Pyq_11
     integer         :: nu, nd, nl, nf
  end type qed_split_mat_nlo

  ! a NNLO splitting matrix (multiplies (alpha/(2pi) )^2) 
  ! contains only Plq splitting!
  type qed_split_mat_nnlo 
     type(grid_conv) :: Plq_02
     integer         :: nu, nd, nl, nf
  end type qed_split_mat_nnlo
\end{lstlisting}     
%
%\commentpn{No paragraph here}
Above, \ttt{y} denotes a photon and the pairs of integers \ttt{01},
\ttt{11} and \ttt{02} denote the orders in the QCD and QED couplings,
respectively. Besides the number of quarks \ttt{nf}, these splitting
matrices also need the number of up-type (\ttt{nu}) and down-type
quarks (\ttt{nd}) separately, and the number of leptons (\ttt{nl}).
%
Note that the splitting functions of order $\alpha$ (i.e.\ \ttt{01})
for the leptons are simply obtained from the ones
involving quarks by adjusting colour factors and couplings.

A call to the subroutine 
\begin{lstlisting}     
  subroutine InitQEDSplitMat(grid, qed_split)
    use qed_splitting_functions
    type(grid_def),      intent(in)    :: grid
    type(qed_split_mat), intent(inout) :: qed_split
    [...]
  end subroutine InitQEDSplitMat
\end{lstlisting}     
initializes the \ttt{qed\_split\_mat} object \ttt{qed\_split} and sets all QED
splitting functions on the given \ttt{grid}.
%
The above QED objects can be used for any sensible value of the
numbers of flavours, on the condition that one first registers
the current number of flavours with a call to
\begin{lstlisting}
  QEDSplitMatSetNf(qed_split, nl, nd, nu)
\end{lstlisting}
where \ttt{nl}, \ttt{nd} and \ttt{nu} are respectively the current
numbers of light leptons, down-type and up-type quarks.
%
In practice, this is always handled internally by the QED-QCD
evolution routines, based on the thresholds encoded in the QED
coupling.\footnote{While the QCD splitting functions are initialised
  and stored separately for each relevant value of $n_f$, in the QED
  case the parts that depend on the numbers of flavours are separated
  out.
  %
  Only when the convolutions with PDFs are performed are the relevant
  $n_f$ and electric charge factors included.}
%
The one situation where a user would need to call this routine
directly is if they wish to manually carry out convolutions of the
QED splitting functions with a PDF.

Subroutines \ttt{Copy} and \ttt{Delete} are also provided for the
\ttt{qed\_split\_mat} type. As in the pure QCD case, convolutions with
QED splitting functions can be represented by the \ttt{.conv.} operator or
using the product sign \ttt{*}.


\myparagraph{PDF arrays with photons and leptons}

A call to the subroutine
\begin{lstlisting}  
  subroutine AllocPDFWithPhoton(grid, pdf)
    type(grid_def), intent(in) :: grid
    real(dp),       pointer    :: pdf(:,:)
    [...]
  end subroutine AllocPDFWithPhoton
\end{lstlisting}
allocates PDFs (\ttt{pdf}) including photons, while a call to
\begin{lstlisting}  
  subroutine AllocPDFWithLeptons(grid, pdf)
    type(grid_def), intent(in) :: grid
    real(dp),       pointer    :: pdf(:,:)
    [...]
    end subroutine AllocPDFWithLeptons
\end{lstlisting}
allocates PDFs including both photons and leptons.
%
The two dimensions of the \ttt{pdf} refer respectively to
the index of the $x$ value in the \ttt{grid}, and to the flavour index.
The flavour indices for photons and leptons %in the arrays \ttt{pdf}
are
given by
\begin{lstlisting}  
  integer, parameter, public :: iflv_photon   =  8
  integer, parameter, public :: iflv_electron =  9
  integer, parameter, public :: iflv_muon     = 10
  integer, parameter, public :: iflv_tau      = 11
\end{lstlisting}
where each \ttt{pdf(:,9:11)} contains the sum of a lepton and
anti-lepton flavour (which are identical). Note that if one were to
extend the calculation of lepton PDFs to higher order in $\alpha$,
then an asymmetry in the lepton and anti-lepton distribution would
arise, due to the \ttt{Plq\_03} splitting function.
%\ak{I think this
%  needs more explanation}.
%
%\gps{Giulia had found a ref that gave a nice explanation, maybe we can
%  add that ref here?}
%\commentpn{What about:
In fact, at that order, there are also graphs with three
electromagnetic vertices on the quark line and three on the lepton line, that change sign if the lepton line is charge-conjugated.\footnote{Analogous terms appear in the non-singlet three-loop splitting functions~\cite{NNLO-NS}.}
%
%
In that case it would become useful
to have separate indices for leptons and anti-leptons.

% \footnote{
%   Were one to use \hoppet to obtain the distribution of
%   partons inside a lepton, then it would become useful to separate
%   lepton and anti-lepton PDFs.
%   % 
%   However further considerations arise, notably for the treatment of
%   the $1-x \ll 1$ region.\commentgz{explain better?} 
%   % 
%   See for example discussion in Ref.~\cite{Frixione:2023gmf}\commentgz{update refs}.}

The subroutine \ttt{AllocPDFWithPhotons} allocates the \ttt{pdf}
array with the flavour index from -6 to 8,
while in the subroutine \ttt{AllocPDFWithLeptons},
the flavour index extends from -6 to 11.


\myparagraph{PDF tables with photons and leptons}

Next one needs to prepare a \ttt{pdf\_table} object forming the
interpolating grid for the evolved PDF's. We recall that the
\ttt{pdf\_table} object contains an underlying array \ttt{pdf\_table\%tab(:,:,:)},
where the first index loops over $x$ values, the second loops over
flavours and the last loops over $Q^2$ values.


This is initialized by a call to 
\begin{lstlisting}
  subroutine AllocPdfTableWithLeptons(grid, pdftable, Qmin, Qmax & 
  & [, dlnlnQ] [, lnlnQ_order] [, freeze_at_Qmin])
   use qed_objects
   type(grid_def),    intent(in)    :: grid
   type(pdf_table),   intent(inout) :: pdftable
   real(dp), intent(in)             :: Qmin, Qmax
   real(dp), intent(in), optional   :: dlnlnQ
   integer,  intent(in), optional   :: lnlnQ_order
   logical,  intent(in), optional   :: freeze_at_Qmin
   [...]
   end subroutine AllocPdfTableWithLeptons
\end{lstlisting}
that is identical to the one without photon or leptons, the only difference is that the
maximum pdf flavour index in \ttt{pdf\_table\%tab} now includes the photon and leptons.
%
An analogous subroutine \ttt{AllocPdfTableWithPhoton} includes the
photon but no leptons.

\myparagraph{Evolution with photons and leptons}


To fill a table via an evolution from an initial scale, one calls the subroutine 
\begin{lstlisting}
    subroutine EvolvePdfTableQED(table, Q0, pdf0, dh, qed_split, &
    &  coupling, coupling_qed, nloop_qcd, nloopqcd_qed, with_Plq_nnloqed)
       type(pdf_table),        intent(inout) :: table
       real(dp),               intent(in)    :: Q0
       real(dp),               intent(in)    :: pdf0(:,:)
       type(dglap_holder),     intent(in)    :: dh
       type(qed_split_mat),    intent(in)    :: qed_split
       type(running_coupling), intent(in)    :: coupling
       type(qed_coupling),     intent(in)    :: coupling_qed
       integer,                intent(in)    :: nloop_qcd, nloopqcd_qed
       logical,  optional,     intent(in)    :: with_Plq_nnloqed
       [...]
    end subroutine EvolvePdfTableQED   
\end{lstlisting}
where \ttt{table} is the output, \ttt{Q0} the initial scale and
\ttt{pdf0} is the PDF at the initial scale.
%
We recall that the lower and upper limits on 
scales in the table are as set at initialisation time for the table.
%
When \ttt{nloopqcd\_qed} is set to 1 (0) mixed QCD-QED effects are
(are not) included in the evolution.
%
Setting the variable \ttt{with\_Plq\_nnloqed=.true.}\ includes also the NNLO $P_{lq}$
splittings in the evolution.

To perform the evolution \ttt{EvolvePdfTableQED} calls the routine 
\begin{lstlisting}
    subroutine QEDQCDEvolvePDF(dh, qed_sm, pdf, coupling_qcd, coupling_qed,&
    &                     Q_init, Q_end, nloop_qcd, nqcdloop_qed, with_Plq_nnloqed)
       type(dglap_holder),     intent(in), target :: dh
       type(qed_split_mat),    intent(in), target :: qed_sm
       type(running_coupling), intent(in), target :: coupling_qcd
       type(qed_coupling),     intent(in), target :: coupling_qed
       real(dp),               intent(inout)      :: pdf(0:,ncompmin:)
       real(dp),               intent(in)         :: Q_init, Q_end
       integer,                intent(in)         :: nloop_qcd
       integer,                intent(in)         :: nqcdloop_qed
       logical,  optional,     intent(in)         :: with_Plq_nnloqed
       [...]
       end subroutine QEDQCDEvolvePDF
\end{lstlisting}
Given the \ttt{pdf} at an initial scale \ttt{Q\_init}, it
evolves it to scale \ttt{Q\_end}, overwriting the \ttt{pdf} array. 
%
In order to get interpolated PDF values from the table we use the
\ttt{EvalPdfTable\_*} calls, described in
\ifreleasenote
Section~7.2 of Ref.~\cite{Salam:2008qg}.
\else
Section~\ref{sec:acc_table}.
\fi
%
However the \texttt{pdf} array that is passed as an argument and that
is set by those subroutines should range not from \texttt{(-6:6)} but
instead from \texttt{(-6:8)} if the PDF just has photons and
\texttt{(-6:11)} if the PDF also includes leptons.\footnote{Index $7$
is a historical artefact associated with internal \hoppet bookkeeping
and should be ignored in the resulting \ttt{pdf} array.}

Note that at the moment, when QED effects are included, cached
evolution is not supported.


\subsection{Streamlined interface with QED effects}
The streamlined interface including QED effects works as in the case of pure QCD evolution.
One has to add the following call
\begin{lstlisting}
  logical use_qed, use_qcd_qed, use_Plq_nnlo
  ...
  call hoppetSetQED(use_qed, use_qcd_qed, use_Plq_nnlo)
\end{lstlisting}
before using the streamlined interface routines.
%
The \ttt{use\_qed} argument turns QED evolution on/off at order
${\cal O}(\alpha)$ (i.e.\ items \ref{item:qed1} and \ref{item:qed2} in
the enumerated list at the beginning of
Sec.~\ref{sec:qed-evolution}).
%
The \ttt{use\_qcd\_qed} one turns
mixed QCD$\times$QED effects on/off in the evolution (i.e.\ item \ref{item:qed3}) and \ttt{use\_Plq\_nnlo} turns 
the order $\alpha^2 P_{\ell q}$ splitting function on/off (i.e.\
item \ref{item:qed4}).
%
Without this call, all QED corrections
are off.

With the above, the streamlined interface can then be used as normal.
%
E.g.\ by calling the \ttt{hoppetEvolve(...)} function to fill the PDF
table and \ttt{hoppetEval(x,Q,f)} to evaluate the PDF at a given $x$
and $Q$.
%
Note that the \ttt{f} array in the latter call must be
suitably large, e.g.\ \ttt{f(-6:11)} for a PDF with leptons.
%
Examples of the streamlined interface being used with QED evolution
can be found in
\begin{quote}
  \repolink{example_f90/tabulation_example_qed_streamlined.f90}{example\_f90/tabulation\_example\_qed\_streamlined.f90}\\
  \repolink{example_f77/cpp_tabulation_example_qed.cc}{example\_f77/cpp\_tabulation\_example\_qed.cc}
\end{quote}
%\gps{added this last paragraph}
%\gps{NB: to make links work, I've hard-coded them to the current
%  branch, but we should switch over to the master/main branch}

%======================================================================
%======================================================================
\section{QCD evolution at N$^3$LO}
\label{sec:n3lo-evolution}
In recent years significant progress in determining the perturbative
components needed for unpolarised QCD evolution at N$^3$LO has been
made, cf.\ \ref{sec:pqcd} for more details on the evolution at N$^3$LO.
%
This has now reached the stage that two PDF groups have released fits
at approximate N$^3$LO (aN$^3$LO)
accuracy~\cite{McGowan:2022nag,NNPDF:2024nan} along with their
combination in Ref.~\cite{Cridge:2024icl}.
%
Of the three contributions that are needed at full N$^3$LO accuracy
only two are fully known.
%
In particular the four-loop
$\beta$-function~\cite{vanRitbergen:1997va,Czakon:2004bu} and
associated mass thresholds~\cite{Chetyrkin:1997sg} have been known for
a very long time.
%
On the other hand the intricate calculations of the three-loop
matching relations needed for both the single and two mass VFNS were
only very recently
completed~\cite{Bierenbaum:2009mv,Ablinger:2010ty,Kawamura:2012cr,Blumlein:2012vq,ABLINGER2014263,Ablinger:2014nga,Ablinger:2014vwa,Behring:2014eya,Ablinger:2019etw,Behring:2021asx,Ablinger:2023ahe,Ablinger:2024xtt}.
%
Finally the four-loop splitting functions entering the DGLAP equation
are currently not known exactly, except for certain $n_f$-dependent
terms~\cite{Gracey:1994nn,Davies:2016jie,Moch:2017uml,Gehrmann:2023cqm,Falcioni:2023tzp,Gehrmann:2023iah}.
%
However, enough Mellin-moments have now been computed that together
with the exact pieces just mentioned and known small- and large-$x$
behaviours, approximate splitting functions suitable for phenomenology
can be reliably
determined~\cite{McGowan:2022nag,NNPDF:2024nan,Moch:2021qrk,Falcioni:2023luc,Falcioni:2023vqq,Moch:2023tdj,Falcioni:2024xyt,Falcioni:2024qpd}.
%
We have therefore incorporated the aforementioned pieces, using the
code that is publicly available with those reference, with the
intention of updating the splitting functions as they become more
precisely known. 

For the specific implementation in \hoppet version 2.0.0 we rely on
the approximations computed in
Refs.~\cite{Davies:2016jie,Moch:2017uml,Falcioni:2023luc,Falcioni:2023vqq,Moch:2023tdj,Falcioni:2024xyt,Falcioni:2024qpd} (FHMPRUVV),
the implementation of the three-loop (single-mass) VFNS as found in
Ref.~\cite{BlumleinCode} which contains code found in
Refs.~\cite{Ablinger:2024xtt,Fael:2022miw} and our own implementation
of the four-loop running coupling.\footnote{The implementation does
not allow for non-standard values of the QCD Casimir invariants beyond
three loops.}
%
Since the implementation here extends core features of \hoppet that
were already available in version 1.1.0, very little is needed on the
side of the user to invoke the evolution.
%
Most importantly all routines that take an \ttt{nloop} argument,
e.g. \ttt{InitDglapHolder}, \ttt{InitRunningCoupling}, and
\ttt{hoppetStart}, now support \ttt{nloop = 4}.
%
The user also has a few choices they can make in terms of the
splitting functions used.
%
The approximate splitting functions can return
three different
choices corresponding to upper and lower edges of an uncertainty band and their
average. 
%
The user can control which one to use through the
\ttt{n3lo\_splitting\_variant} which takes the values
\ttt{n3lo\_splitting\_Nfitav} (default),
\ttt{n3lo\_splitting\_Nfiterr1}, or \ttt{n3lo\_splitting\_Nfiterr2}.
%
When the exact splitting functions become available they will
correspond to \ttt{n3lo\_splitting\_exact}.
%
Similarly if a set of parametrisations of the exact splitting
functions become available they will correspond to
\ttt{n3lo\_splitting\_param}.

Since it is expected that more precise approximations will become
available in the future we have also implemented a flag,
\ttt{n3lo\_splitting\_approximation}, in
\masterlink{src/dglap\_choices.f90} that controls which approximations
to use. By default it uses the most recent set of approximations by
the FHMPRUVV group. At the time of writing it takes three values: the
older implementation
\ttt{n3lo\_splitting\_approximation\_up\_to\_2310\_05744} meaning the
approximations in papers up to and including Ref.~\cite{Moch:2023tdj};
\ttt{n3lo\_splitting\_approximation\_up\_to\_2404\_09701} meaning the
approximations in papers up to and including
Ref.~\cite{Falcioni:2024xyt} and the default value of
\ttt{n3lo\_splitting\_approximation\_up\_to\_2410\_08089} meaning the
approximations in papers up to and including
Ref.~\cite{Falcioni:2024qpd}.

Besides the splitting functions, we have also incorporated the
single-mass thresholds up to N$^3$LO, as calculated in
Refs.~\cite{Bierenbaum:2009mv,Ablinger:2010ty,Kawamura:2012cr,Blumlein:2012vq,ABLINGER2014263,Ablinger:2014nga,Ablinger:2014vwa,Behring:2014eya,Ablinger:2019etw,Behring:2021asx,Ablinger:2023ahe,Ablinger:2024xtt}
and kindly supplied to us in the form of Fortran
code~\cite{BlumleinCode,Fael:2022miw}.
%
Note that because many of these contributions are available only in
exact form (i.e.\ using extensive \texttt{hplog5} evaluations), they
involve a significant initialisation time, of the order a
$10{-}30\,\text{s}$.\footnote{As and when faster 
  parametrised versions become available, we will incorporate them.}


In Table~\ref{tab:n3lo-evolve} we show the results of the full N$^3$LO
evolution in the VFNS, using the most up-to-date perturbative input at
the time of writing this release note. To assess the evolution we take
the initial condition of Ref.~\cite{Dittmar:2005ed} at an initial
scale $\sqrt{2}~\text{GeV}$ and $n_f=3$. We evolve to
$Q=100~\text{GeV}$. The table can be reproduced by running
\repolink{example_f90/tabulation_example_n3lo.f90}{example\_f90/tabulation\_example\_n3lo.f90}. The
numbers are obtained using parametrised NNLO splitting functions, but
exact mass thresholds at this order.
%
The table has been generated with $\texttt{dy}=0.05$,
$\texttt{dlnlnQ}=\text{dy/4}$.
%
Increasing $\texttt{dy}=0.10$ leaves the results unchanged at the
precision shown, while going to $\texttt{dy}=0.20$ for higher speed
would change the results by a relative amount below $10^{-4}$.

Table~\ref{tab:n3lo-evolve} cannot be directly compared to the
benchmarking tables of Ref.~\cite{Cooper-Sarkar:2024crx} because they
do not include the mass thresholds in the N3LO evolution.
%
To facilitate comparisons we therefore additionally provide
Table~\ref{tab:n3lo-evolve-nf4}, correspond to fixed-flavour ($n_f=4$)
evolution, with
\ttt{n3lo\_splitting\_approximation\_up\_to\_2404\_09701}.
%
Of the MSHT and NNPDF results, the NNPDF results (Table~2 of
Ref.~\cite{Cooper-Sarkar:2024crx}) are closer to ours.



\begin{table}[ht!]
  %  \footnotesize \centering
  \small \centering
  \begin{tabular}{c|cccccccc}
    $x$ &  $u-\bar u$ &$d-\bar d$ &$\bar d-\bar u$ &  $ 2(\bar u+\bar d)$  &$s+\bar s$&$c+\bar c$ &$b+\bar b$  &   $g$\\
    \toprule
$10^{-7}$ &  $1.0589^{-4}$ &  $4.8664^{-5}$ &  $8.1967^{-6}$ &  $1.6234^{+2}$ &  $8.0100^{+1}$ &  $7.7207^{+1}$ &  $6.5254^{+1}$ &  $1.1238^{+3}$ \\
$10^{-6}$ &  $5.9691^{-4}$ &  $3.2634^{-4}$ &  $3.3146^{-5}$ &  $7.6786^{+1}$ &  $3.7544^{+1}$ &  $3.5836^{+1}$ &  $2.9889^{+1}$ &  $5.1159^{+2}$ \\
$10^{-5}$ &  $3.0235^{-3}$ &  $1.7532^{-3}$ &  $1.3081^{-4}$ &  $3.5436^{+1}$ &  $1.7044^{+1}$ &  $1.6106^{+1}$ &  $1.3142^{+1}$ &  $2.2224^{+2}$ \\
$10^{-4}$ &  $1.4079^{-2}$ &  $8.2354^{-3}$ &  $4.9511^{-4}$ &  $1.5611^{+1}$ &  $7.2731^{+0}$ &  $6.7862^{+0}$ &  $5.3294^{+0}$ &  $8.8594^{+1}$ \\
$10^{-3}$ &  $6.0849^{-2}$ &  $3.5086^{-2}$ &  $1.7751^{-3}$ &  $6.3823^{+0}$ &  $2.7798^{+0}$ &  $2.5204^{+0}$ &  $1.8516^{+0}$ &  $3.0349^{+1}$ \\
$10^{-2}$ &  $2.3361^{-1}$ &  $1.3074^{-1}$ &  $5.8324^{-3}$ &  $2.2673^{+0}$ &  $8.5415^{-1}$ &  $7.0444^{-1}$ &  $4.6228^{-1}$ &  $7.7859^{+0}$ \\
$0.1$    &  $5.4846^{-1}$ &  $2.6950^{-1}$ &  $9.9965^{-3}$ &  $3.8453^{-1}$ &  $1.1248^{-1}$ &  $6.8296^{-2}$ &  $3.7899^{-2}$ &  $8.4964^{-1}$ \\
$0.3$    &  $3.4441^{-1}$ &  $1.2761^{-1}$ &  $2.9457^{-3}$ &  $3.4575^{-2}$ &  $8.8873^{-3}$ &  $3.9659^{-3}$ &  $2.0846^{-3}$ &  $7.8697^{-2}$ \\
$0.5$    &  $1.1790^{-1}$ &  $3.0597^{-2}$ &  $3.6526^{-4}$ &  $2.3206^{-3}$ &  $5.6808^{-4}$ &  $2.0185^{-4}$ &  $1.1382^{-4}$ &  $7.6337^{-3}$ \\
$0.7$    &  $1.9329^{-2}$ &  $2.9648^{-3}$ &  $1.2848^{-5}$ &  $5.2429^{-5}$ &  $1.2662^{-5}$ &  $3.4020^{-6}$ &  $2.4956^{-6}$ &  $3.7094^{-4}$ \\
$0.9$    &  $3.3153^{-4}$ &  $1.6737^{-5}$ &  $8.0961^{-9}$ &  $2.5214^{-8}$ &  $6.6432^{-9}$ &  $7.6153^{-10}$ &  $1.4323^{-9}$ &  $1.1716^{-6}$ 
  \end{tabular}
  \caption{N$^3$LO evolution of the initial condition given in Section
    4.4 of Ref.~\cite{Dittmar:2005ed}, using the same notation where
    $a\cdot10^{b} = a^b$. The evolution is performed taking the
    initial condition at $\sqrt{2}~\text{GeV}$ (just below the charm
    mass) and evolving in the VFNS up to $Q=100~\text{GeV}$, with a
    charm-quark pole mass of $\sqrt{2}+10^{-9}\GeV$ and a bottom-quark
    pole mass of $4.5\GeV$.
    %
    The NNLO splitting functions are the parametrised form
    (\texttt{nnlo\_splitting\_variant = nnlo\_splitting\_param}), to 
    facilitate comparisons by other groups, and the N3LO splitting
    functions use the
    \ttt{n3lo\_splitting\_approximation = n3lo\_splitting\_approximation\_up\_to\_2410\_08089} choice.}
  \label{tab:n3lo-evolve}
\end{table}

\begin{table}
  \small \centering
  \begin{tabular}{c|cccccccc}
    $x$ &  $u-\bar u$ &$d-\bar d$ &$\bar d-\bar u$ &  $ 2(\bar u+\bar d)$  &$s-\bar s$ &$s+\bar s$&$c+\bar c$ &   $g$\\
    \toprule
$10^{-7}$ &  $9.8370^{-5}$ &  $4.5171^{-5}$ &  $7.5013^{-6}$ &  $1.4888^{+2}$ &  $-2.9105^{-5}$ &             $7.3368^{+1}$ &  $7.2653^{+1}$ &  $1.0851^{+3}$ \\
$10^{-6}$ &  $5.6405^{-4}$ &  $3.0895^{-4}$ &  $3.0730^{-5}$ &  $7.1925^{+1}$ &  $-4.6739^{-5}$ &             $3.5111^{+1}$ &  $3.4544^{+1}$ &  $5.0392^{+2}$ \\
$10^{-5}$ &  $2.8946^{-3}$ &  $1.6810^{-3}$ &  $1.2302^{-4}$ &  $3.3868^{+1}$ &  $-3.5766^{-6}$ &             $1.6258^{+1}$ &  $1.5808^{+1}$ &  $2.2292^{+2}$ \\
$10^{-4}$ &  $1.3633^{-2}$ &  $7.9832^{-3}$ &  $4.7274^{-4}$ &  $1.5188^{+1}$ &  $\phantom{-}2.1123^{-4}$ &   $7.0599^{+0}$ &  $6.7033^{+0}$ &  $9.0268^{+1}$ \\
$10^{-3}$ &  $5.9567^{-2}$ &  $3.4382^{-2}$ &  $1.7232^{-3}$ &  $6.3028^{+0}$ &  $\phantom{-}3.9314^{-4}$ &   $2.7387^{+0}$ &  $2.4621^{+0}$ &  $3.1350^{+1}$ \\
$10^{-2}$ &  $2.3130^{-1}$ &  $1.2962^{-1}$ &  $5.7645^{-3}$ &  $2.2675^{+0}$ &  $-1.9644^{-4}$ &             $8.5255^{-1}$ &  $6.6402^{-1}$ &  $8.1568^{+0}$ \\
$0.1$    &  $5.5131^{-1}$ &  $2.7140^{-1}$ &  $1.0085^{-2}$ &  $3.8980^{-1}$ &  $-3.1812^{-4}$ &             $1.1388^{-1}$ &  $5.9843^{-2}$ &  $9.0615^{-1}$ \\
$0.3$    &  $3.5044^{-1}$ &  $1.3015^{-1}$ &  $3.0145^{-3}$ &  $3.5426^{-2}$ &  $-3.8409^{-5}$ &             $9.0900^{-3}$ &  $3.3507^{-3}$ &  $8.4431^{-2}$ \\
$0.5$    &  $1.2112^{-1}$ &  $3.1518^{-2}$ &  $3.7779^{-4}$ &  $2.3973^{-3}$ &  $-3.3053^{-6}$ &             $5.8501^{-4}$ &  $1.7709^{-4}$ &  $8.1568^{-3}$ \\
$0.7$    &  $2.0078^{-2}$ &  $3.0889^{-3}$ &  $1.3448^{-5}$ &  $5.4598^{-5}$ &  $-1.1810^{-8}$ &             $1.3105^{-5}$ &  $3.6963^{-6}$ &  $3.9248^{-4}$ \\
$0.9$    &  $3.5128^{-4}$ &  $1.7793^{-5}$ &  $8.6472^{-9}$ &  $2.6378^{-8}$ &  $-1.5848^{-10}$ &            $6.8222^{-9}$ &  $2.6776^{-9}$ &  $1.2262^{-6}$ 
  \end{tabular}
  \caption{N$^3$LO evolution of the initial condition given in Section
    4.4 of Ref.~\cite{Dittmar:2005ed}, using the same notation where
    $a\cdot10^{b} = a^b$. The evolution is performed taking the
    initial condition at $\sqrt{2}~\text{GeV}$ (just below the charm
    mass) and evolving in the FFN scheme ($n_f = 4$) up to
    $Q=100~\text{GeV}$.
    %
    The NNLO splitting functions are the parametrised form
    (\texttt{nnlo\_splitting\_variant = nnlo\_splitting\_param}), to 
    facilitate comparisons by other groups, and the N3LO splitting
    functions use the
    \ttt{n3lo\_splitting\_approximation = n3lo\_splitting\_approximation\_up\_to\_2310\_05744} choice. This table can be directly compared to tables 1 and 2 in Ref.~\cite{Cooper-Sarkar:2024crx}}
  \label{tab:n3lo-evolve-nf4}
\end{table}

%\ak{Started collecting some references here as there are many and they are not that easy to find...}
%\begin{itemize}
%\item As of 2024-04-26 this is the list of all papers on 4-loop splitting functions~\cite{Davies:2016jie,Moch:2017uml,Moch:2021qrk,Falcioni:2023luc,Falcioni:2023vqq,Gehrmann:2023cqm,Falcioni:2023tzp,Moch:2023tdj,Gehrmann:2023iah,Falcioni:2024xyt} and this is the list of what is actually implemented~\cite{Davies:2016jie,Moch:2017uml,Falcioni:2023luc,Falcioni:2023vqq,Moch:2023tdj,Falcioni:2024xyt}
%\item And this is the list of all papers relating to the 3-loop mass thresholds~\cite{Bierenbaum:2009mv,Ablinger:2010ty,Kawamura:2012cr,Blumlein:2012vq,ABLINGER2014263,Ablinger:2014nga,Ablinger:2014vwa,Behring:2014eya,Ablinger:2019etw,Behring:2021asx,Ablinger:2023ahe,Ablinger:2024xtt}. The code that we use relies on this set~\cite{Ablinger:2010ty,Blumlein:2012vq,ABLINGER2014263,Ablinger:2014nga,Ablinger:2014vwa,Behring:2014eya,Ablinger:2019etw,Behring:2021asx,Ablinger:2023ahe,Ablinger:2024xtt} plus the code from Bluemlein for which we do not yet have a reference.
%\end{itemize}


%======================================================================
%======================================================================
\section{Hadronic Structure Functions}
\label{sec:structure-funcs}
As of \hoppet version 2.0.0 the code provides access to the massless
hadronic structure functions. The structure functions are expressed as
convolutions of a set of massless hard coefficient functions and PDFs,
and make use of the tabulated PDFs and streamlined interface.  They
are provided such that they can be used directly for cross section
computations in DIS or VBF, as implemented in {\tt
  disorder}~\cite{Karlberg:2024hnl} and the {\tt proVBFH}
package~\cite{Cacciari:2015jma,Dreyer:2016oyx,Dreyer:2018qbw,Dreyer:2018rfu}.

The massless structure functions have been found to be in good
agreement with those that can be obtained with
\APFELPP{}~\cite{Bertone:2013vaa,Bertone:2017gds} (at the level of
$10^{-5}$ relative precision).
%
The benchmarks with \APFELPP{} and the code used to carry them out are
described in detail in Ref.~\cite{Bertone:2024dpm} and at
\url{https://github.com/alexanderkarlberg/n3lo-structure-function-benchmarks}. A
simplified version of that benchmark is also included in the \hoppet{}
repository and can be found in
\masterlink{benchmarking/structure\_functions\_benchmark\_checks.f90}.
%
Technical details on the implementation of the structure functions in
\hoppet can be found in
Refs.~\cite{Dreyer:2016vbc,Karlberg:2016zik,Bertone:2024dpm}.
%A small sample program
%that prints out the structure functions can be found in
%\masterlink{example\_f90/structure\_functions\_example.f90}.

The structure functions have been implemented including only QCD
corrections up to N$^3$LO using both the exact and parametrised
coefficient functions found in
Refs.~\cite{vanNeerven:1999ca,vanNeerven:2000uj,Moch:2004xu,Vermaseren:2005qc,Moch:2008fj,Davies:2016ruz,Blumlein:2022gpp},
and can make use of PDFs evolved at N$^3$LO as described in
Sec.~\ref{sec:n3lo-evolution}. The structure functions can also be
computed using PDFs obtained through LHAPDF~\cite{LHAPDF}.

\subsection{Initialisation}
\label{sec:structure-funcs-init}

The structure functions can be accessed from inside the module
\ttt{structure\_functions}. They can also be accessed through the
streamlined interface by prefixing \ttt{hoppet} as described in
section~\ref{sec:structure-functions-streamlined}.
%
The description below corresponds to a high-level interface, which
relies on elements such as the \ttt{grid} and splitting functions
having been initialised in the streamlined interface, through a call
to \ttt{hoppetStart} or \ttt{hoppetStartExtended}, cf.\
\ifreleasenote
Section~8 of Ref.~\cite{Salam:2008qg}.\footnote{Users needing a lower level
  interface should inspect the code in \masterlink{src/structure\_functions.f90}.}
\else
Section~\ref{sec:vanilla}.\footnote{Users needing a lower level
  interface should inspect the code in \masterlink{src/structure\_functions.f90}.}
\fi
%
After this initialisation has been carried out one calls
\begin{lstlisting}
  call StartStrFct(order_max [, nflav] [, scale_choice] &
                  & [, constant_mu] [ ,param_coefs] [ ,wmass] [ ,zmass])
\end{lstlisting}
specifying as a minimum the perturbative order --- currently
\ttt{order\_max} $ \le 4$ (\ttt{order\_max} $ =1$ corresponds to LO).

If \ttt{nflav} is not passed as an argument, the structure functions
are initialised to support a variable flavour-number scheme (the
masses that are used at any given stage will be those set in the
streamlined interface).
%
Otherwise a fixed number of light flavours is used, as indicated by
\ttt{nflav}, which speeds up initialisation.
%
Note that specifying a variable flavour-number scheme only has an
impact on the evolution and on $n_f$ terms in the coefficient
functions.
%
The latter, however always assume massless quarks.
%
Hence in both the fixed and variable flavour-number scheme the
structure functions should not be considered reliable if  $Q$ is comparable to the quark mass.
%structure functions should only be considered reliable for $Q \gg m$.
%\commentpn{Actually this is also true in a fixed flavour number
%  scheme, as long as mass effects are not being included in the
%  coefficient functions, i.e. for $Q\gg m$ variable flavour is the
%  same as fixed flavours (including the flavour of mass $m$ as a light
%  flavour.)}\ak{I added a mentioning of the fixed flavour
%  limitations as well}
%\ak{Need to adapt this to what the code acutally does}
Together \ttt{xR}, \ttt{xF}, \ttt{scale\_choice}, and \ttt{constant\_mu} control
the renormalisation and factorisation scales and the degree of
flexibility that will be available in choosing them at later stages.
%
Specifically the (integer) \ttt{scale\_choice}
argument can take several values:
\begin{itemize}
\item \ttt{scale\_choice\_Q} (default) means that the code will always
  use $Q$ multiplied by \ttt{xR} or \ttt{xF} as the renormalisation
  and factorisation scale respectively (with \ttt{xR} or \ttt{xF} as
  set at initialisation).
\item \ttt{scale\_choice\_fixed} corresponds to a fixed scale
  \ttt{constant\_mu}, multiplied by \ttt{xR} or \ttt{xF} as set at
  initialisation.
\item \ttt{scale\_choice\_arbitrary} allows the user to choose
  arbitrary scales at the moment of evaluating the structure
  functions.
  %
  In this last case, the structure functions are saved as separate arrays, 
  one for each perturbative order, and with dedicated additional arrays
  for terms proportional to logarithms of $Q/\mu_F$. 
  % 
  This makes for a slower evaluation compared to the two other
  scale choices.
\end{itemize}
%
% The default value for \ttt{scale\_choice} is \ttt{scale\_choice\_Q} which means that $Q$
% multiplied by \ttt{xR} or \ttt{xF} is used as the renormalisation or
% factorisation scale respectively.
% %
% The choice \ttt{scale\_choice\_fixed} corresponds to a fixed scale
% \ttt{constant\_mu} (multiplied by \ttt{xR} or \ttt{xF}).
% %
% Should a user wish to use some arbitrary scale choice,
% \ttt{scale\_choice} should be set to \ttt{scale\_choice\_arbitrary}.
% %
% \gps{Do we want to introduce some constants such as
%   \ttt{scale\_choice\_Q=1}, \ttt{scale\_choice\_fixed=0}, \ttt{scale\_choice\_arbitrary=2}?} \ak{Yes, and done!}
% %
% In this last case, the structure functions are saved as arrays not only in
% $Q$ but also $\mu_R$ and $\mu_F$.
% %
% This makes for a slightly slower evaluation compared to the two other
% scale choices.

If \ttt{param\_coefs} is set to \ttt{.true.}\ (its default) then the
structure functions are computed using the NNLO and N$^3$LO
parametrisations found in
Refs.~\cite{vanNeerven:1999ca,vanNeerven:2000uj,Moch:2004xu,Vermaseren:2005qc,Moch:2008fj,Davies:2016ruz},
which are stated to have a relative precision of a few permille (order
by order) except at particularly small or large values of $x$.
%
Otherwise
the exact versions are used.\footnote{The LO and NLO coefficient
functions are always exact as their expressions are very
compact.}
%
This however means that the initialisation becomes slow (about two
minutes rather than a few seconds).
%
Given the good accuracy of the parametrised coefficient functions,
they are to be preferred for most applications.
% 
% and since the parametrised
% expressions are good to a relative accuracy of $10^{-4}$ for most
% values of $x$, it is recommended to use the parametrised option for
% most applications.
% 
Note that the exact expressions also add to compilation time and need
to be explicitly enabled with the \texttt{-DHOPPET\_USE\_EXACT\_COEF=ON}
CMake flag (or the \ttt{--enable-exact-coefs} configure
flag).\footnote{At N$^3$LO they rely on an extended version of
  \ttt{hplog}~\cite{FortranPolyLog}, \texttt{hplog5} version 1.0, that
  is able to handle harmonic polylogarithms up to weight 5.}

The masses of the electroweak vector bosons are used only to calculate
the weak mixing angle, $\sin^2 \theta_W = 1 - (m_W/m_Z)^2$, which
enters in the neutral-current structure functions.

At this point all the tables that are needed for the structure
functions have been allocated.
%
In order to fill the tables, one first needs to set up the running
coupling and evolve the initial PDF with \ttt{hoppetEvolve}, as
described in
\ifreleasenote
Section~8.2 of Ref.~\cite{Salam:2008qg}.
\else
Section~\ref{sec:vanilla_usage}.
\fi

%
% Care should be taken here such that both the coupling and PDF
% evolution are carried out at the correct perturbative order and with
% mass thresholds as appropriate.
%
With the PDF table filled in the streamlined interface one calls
\begin{lstlisting}
  call InitStrFct(order[, separate_orders] [, xR] [, xF] [, flavour_decomposition])
\end{lstlisting}
specifying the order at which one would like to compute the structure
functions.
%
The logical flag \ttt{separate\_orders} should be set to \ttt{.true.}\ if one
wants access to the individual coefficients of the perturbative
expansion as well as the sum up to some maximum order, \ttt{order}.
%
With \ttt{scale\_choice\_Q} and \ttt{scale\_choice\_fixed}, the
default of \ttt{.false.}\ causes only the sum over perturbative orders
to be stored.
%
This gives faster evaluations of structure functions because it is
only necessary to interpolate the sum over orders, rather than
interpolate one table for each order.
%
With \texttt{scale\_choice\_arbitrary}, the default is \ttt{.true.},
which is the only allowed option, because separate tables for each
order are required for the underlying calculations.

Finally, the optional flag \ttt{flavour\_decomposition} controls an
experimental feature of giving access to the structure functions
decomposed into their underlying quark flavours without the associated
vector boson couplings. It is currently only possible to access the
structure functions in this way up to NLO, and since the feature is
not fully mature we invite interested readers to inspect the source
code directly for more information.

\subsection{Accessing the Structure Functions}
\label{sec:structure-funcs-access}
At this point the structure functions can be accessed as in the following example
\begin{lstlisting}
  real(dp) :: ff(-6:7), x, Q, muR, muF
  
  call StartStrFct(order_max = 4, scale_choice = scale_choice_Q)
  [...]
  call InitStrFct(order_max = 4)
  ff = StrFct(x, Q[, muR] [, muF])
\end{lstlisting}
at the value $x$ and $Q$.
%\gps{Added explicit scale choice arg in init}\ak{Not sure I understand, that's not the signature of the routine}
%
With \ttt{scale\_choice\_arbitrary}, the \ttt{muR} and \ttt{muF}
arguments must be provided.
%
With other scale choices, they do not need to be provided, but if they
are then they should be consistent with the original scale choice.
%
The structure functions in this example are stored in the
array \ttt{ff}. The components of this array can be accessed through
the indices
\begin{lstlisting}
  integer, parameter :: iF1Wp = 1   
  integer, parameter :: iF2Wp = 2   
  integer, parameter :: iF3Wp = 3   
  integer, parameter :: iF1Wm =-1   
  integer, parameter :: iF2Wm =-2   
  integer, parameter :: iF3Wm =-3   
  integer, parameter :: iF1Z  = 4   
  integer, parameter :: iF2Z  = 5   
  integer, parameter :: iF3Z  = 6   
  integer, parameter :: iF1EM =-4   
  integer, parameter :: iF2EM =-5   
  integer, parameter :: iF1gZ = 0  
  integer, parameter :: iF2gZ =-6 
  integer, parameter :: iF3gZ = 7  
\end{lstlisting}
For instance one would access the electromagnetic $F_1$ structure
function through \ttt{ff(iF1EM)}. It is returned at the \ttt{order\_max}
that was specified in \ttt{InitStrFct}.
%
The structure functions can also be accessed order by order if the
\ttt{separate\_orders} flag was set to \ttt{.true.} when initialising.
%
They are then obtained as follows
\begin{lstlisting}
  real(dp) :: flo(-6:7), fnlo(-6:7), fnnlo(-6:7), fn3lo(-6:7), x, Q, muR, muF
  [...]
  call InitStrFct(4, .true.)
  flo   = F_LO(x, Q, muR, muF)
  fnlo  = F_NLO(x, Q, muR, muF)
  fnnlo = F_NNLO(x, Q, muR, muF)
  fn3lo = F_N3LO(x, Q, muR, muF)
\end{lstlisting}
The functions return the individual contributions at each order in
$\as$, including the relevant factor of $\as^n$.
%
Hence the sum of \ttt{flo}, \ttt{fnlo}, \ttt{fnnlo}, and
\ttt{fn3lo} would return the full structure function at N$^3$LO as
contained in \ttt{ff} in the example above.
%
Note that in the \ttt{F\_LO} etc.\ calls, the \ttt{muR} and \ttt{muF}
arguments are not optional and that when a prior scale choice has been
made (e.g. \ttt{scale\_choice\_Q}) they are assumed to be consistent
with that prior scale choice.

An example of structure function evaluations using the Fortran~90
interface is to be found in
\repolink{example_f90/structure_functions_example.f90}{example\_f90/structure\_functions\_example.f90}. 
%%
%\gps{added this link}\ak{It is also at the beginning of the section}

\subsection{Streamlined interface}
\label{sec:structure-functions-streamlined}
The structure functions can also be accessed through the streamlined
interface, so that they may be called for instance from C/\CPP{}. The
functions to be called are very similar to those described above. In
particular a user should call either
\begin{lstlisting}
  call hoppetStartStrFct(order_max)
\end{lstlisting}
with \ttt{order\_max} the maximal
order in $\as$ or, alternatively, the extended version of the interface
\ttt{hoppetStartStrFctExtended} which takes all the same arguments as
\ttt{StartStrFct} described above. One difference is that in order to
use a variable flavour scheme the user should set \ttt{nflav} to a
negative value. After evolving or reading in a PDF, the user then calls
\begin{lstlisting}
  call hoppetInitStrFct(order, separate_orders)
\end{lstlisting}
to initialise the actual structure functions. The structure functions
can then be accessed through the subroutines
\begin{lstlisting}
  real(dp) :: ff(-6:7), flo(-6:7), fnlo(-6:7), fnnlo(-6:7), fn3lo(-6:7), x, Q, muR, muF
  [...]
  call hoppetStrFct(x, Q, muR, muF, ff)        ! Full structure function
  ! or instead, without muR and muF
  call hoppetStrFctNoMu(x, Q, ff)              ! Full structure function
  call hoppetStrFctLO(x, Q, muR, muF, flo)     ! LO term
  call hoppetStrFctNLO(x, Q, muR, muF, fnlo)   ! NLO term
  call hoppetStrFctNNLO(x, Q, muR, muF, fnnlo) ! NNLO term
  call hoppetStrFctN3LO(x, Q, muR, muF, fn3lo) ! N3LO term
\end{lstlisting}
The \CPP{} header contains indices for the structure functions and scale
choices, which are all in the \ttt{hoppet} namespace.
%
\begin{lstlisting}
  const int iF1Wp = 1+6;
  const int iF2Wp = 2+6;
  const int iF3Wp = 3+6;
  const int iF1Wm =-1+6;
  const int iF2Wm =-2+6;
  const int iF3Wm =-3+6;
  const int iF1Z  = 4+6;
  const int iF2Z  = 5+6;
  const int iF3Z  = 6+6;
  const int iF1EM =-4+6;
  const int iF2EM =-5+6;
  const int iF1gZ = 0+6;
  const int iF2gZ =-6+6;
  const int iF3gZ = 7+6;

  const int scale_choice_fixed     = 0;
  const int scale_choice_Q         = 1;
  const int scale_choice_arbitrary = 2;
\end{lstlisting}
Note that in \CPP{} the structure function indices start from 0 and that the \CPP{}
array that is to be passed to functions such as \ttt{hoppetStrFct}
would be defined as \ttt{double ff[14]}.

An example of structure function evaluations using the \CPP{} version of
the streamlined interface is to be found in
\repolink{example_f77/cpp_structure_functions_example.cc}{example\_f77/cpp\_structure\_functions\_example.cc}.
%
%\gps{added this link}

\section{Updated performance studies}
In this section we present some updated performance studies relative
to
%
\ifreleasenote
Section~9 of Ref.~\cite{Salam:2008qg}.
\else
Section~\ref{sec:benchmarks}.
\fi
%
Most importantly we benchmark the accuracy and timing of the new
N$^3$LO evolution. We also present updated plots at NNLO, to give a
better idea of the performance of \hoppet{} on typical compilers and
hardware in 2025.

\begin{figure}[htbp]
    \centering
    \begin{subfigure}{0.49\textwidth}
        \centering
        \includegraphics[width=\textwidth,page=2]{hoppet-v2-accuracy-speed-v-dy.pdf}
        \caption{}
        \label{fig:sub1}
    \end{subfigure}
    \hfill
    \begin{subfigure}{0.49\textwidth}
        \centering
        \includegraphics[width=\textwidth,page=3]{hoppet-v2-accuracy-speed-v-dy.pdf}
        \caption{}
        \label{fig:sub2}
    \end{subfigure}
    \caption{Accuracy (top) and timing (bottom) versus \texttt{dy} at
      (a) NNLO and (b) N$^3$LO in \hoppet{} v2.0.0. The timings were
      obtained on an \texttt{M2Pro} with \texttt{gfortran v14.2} and
      \texttt{-O3} optimisation. We have further used \texttt{ymax = 12} and
      \texttt{dlnlnQ = dy/4}.}
    \label{fig:main}
\end{figure}

The studies are performed using the initial condition of
Ref.~\cite{Dittmar:2005ed} as detailed in
%
\ifreleasenote
Section~9.1 of Ref.~\cite{Salam:2008qg},
\else
Section~\ref{sec:Accuracy},
\fi
%
evolved in the VFNS scheme. At NNLO we use the parametrised splitting
functions and mass thresholds, as per \hoppet defaults.
%
To
assess the accuracy we first create a reference run with a very high
density grid. We then run \hoppet{} with different values of grid
spacing \texttt{dy} keeping \texttt{dlnlnQ = dy/4}. The accuracy is
then computed by looking at the largest relative deviation from the
reference run across either all flavours (all-flav) or the light
flavours (guds).
%
The full tabulation covers the range $10^{-5} < x < 1$,
$2 < Q < 10^4\, \text{GeV}$.\footnote{For tabulations extending to
  significantly smaller $x$ values, it can be advantageous to take a
  smaller \texttt{dlnlnQ} choice, e.g.\ \texttt{dlnlnQ = dy/8},
  because of the steep $\ln Q$ derivative of the parton
  distributions at the smallest $x$ values.}
%
At very large $x$ values, the numerical precision degrades because the
parton distribution functions become a very steep function of $\ln
x$.
%
However, the parton distributions have small values there and so we
carry out our precision study with two potential upper limits on the
$x$ value being probed, $x < 0.9$ and $x < 0.7$.
%
We also exclude PDF flavours in the $x$ and $Q$ vicinity of any sign
change, as per Ref.\cite{Salam:2008qg}.
% since the grids quickly
% deteriorate as $x \to 1$ due to divergent logarithmic derivatives we
% limit ourselves to $x < 0.9$ and $x < 0.7$ which cover any potential
% phenomenological application \ak{is this true?}.

The result for the accuracy study can be seen in the top panels of
Figures~\ref{fig:sub1}--\ref{fig:sub2} at NNLO and N$^3$LO
respectively, as a function of \texttt{dy}.
%
At NNLO, the accuracy comes out similar to previous
versions of \hoppet{.}
%
In particular we note that with \texttt{dy =
  0.2} one obtains a relative accuracy of $10^{-4}$ across all
flavours in the range $x < 0.9$.
%
At the finest grid spacing \texttt{dy = 0.04} a relative accuracy of
$10^{-7}$ can be achieved, good enough for precise benchmark
comparisons as were for instance carried out in
Refs.~\cite{Dittmar:2005ed,Bertone:2024dpm}. We note that in contrast
the recent benchmark of aN$^3$LO codes in
Ref.~\cite{Cooper-Sarkar:2024crx} reaches a relative precision of a
few times $10^{-4}$ at best (cf.\ the gluon PDF at $x = 10^{-2}$ in
Table~\ref{tab:n3lo-evolve-nf4} and Table 1 of
Ref.~\cite{Cooper-Sarkar:2024crx} which differ by $\sim 8\cdot 10^{-4}$.)
%
The scaling of the precision is roughly consistent with a power law in
\texttt{dy}.
%
In particular the Runge-Kutta algorithm for the $Q^2$ evolution is
expected to yield an error proportional to $\texttt{dlnlnQ}^4$, which
translates into a behaviour $\sim \texttt{dy}^4$ in
Fig.~\ref{fig:main}.
%
The observed scaling is, if anything, slightly better than this, given
the factor of $1000$ improvement in accuracy when reducing \texttt{dy}
by a factor of $5$ from $0.2$ to $0.04$.
%
At N$^3$LO we observe a similar level of accuracy as at
NNLO.\footnote{Note that at N3LO we have omitted the 
  all-flav, $x<0.9$ line.
  %
  For that case, we have observed issues affecting the accuracy at the
  level of up to relative $10^{-5}$ accuracy.
  %
  The deterioration is there only when the N3LO mass thresholds are
  used and we suspect that it may be connected with difficulties in the
  integration over the $x \to 1$ region of the mass-threshold
  functions.
  %
  Given that the corresponding preliminary code has been kindly
  provided to us ahead of publication~\cite{BlumleinCode}, we defer
  more detailed study of this matter until the final code is ready.
  %
  We anticipate that it may be possible to address the problem by
  switching to an expansion in powers (and logarithms) of $(1-x)$ for
  $x \to 1$.  }


% for the all-flav, $x<0.9$ case with \texttt{dy} below \texttt{0.1}. We
% have confirmed that this deterioration is due to small numerical
% instabilities in the exact mass threshold expressions at N$^3$LO,
% presumably when $x \to 1$.
% %
% We
% therefore suspect that this could be improved by switching to as
% expansion in powers (and logarithms) of $(1-x)$ when $x \to
% 1$.\footnote{Once the final results from \cite{BlumleinCode} are
%   available we will revisit this question.}
% %
% Since this deterioration is only present at the relative $10^{-5}$
% level, and at very large values of $x$, it is unlikely to be visible
% in any realistic application.
% %
% In particular we note that for
% $x<0.7$ the accuracy is comparable to that at NNLO.

For the timing studies we again run \hoppet{} for different values of
\texttt{dy}, on an \texttt{M2Pro} with \texttt{gfortran v14.2} and
\texttt{-O3} optimisation.
%In order to avoid fluctuations we take the
%average over \ak{N} runs.
%
As discussed in \cite{Salam:2008qg}, the time spent in \hoppet for a
given analysis can be expressed as follows, depending on whether or
not one carries out cached evolution (pre-evolution):
\begin{subequations}
  \label{eq:timing-v2}
  \begin{align}
    t_\text{no pre-ev}   &= t_s + n_\alpha t_\alpha + n_i (t_i  + n_{xQ}\, t_{xQ})\,,\\
    t_\text{with pre-ev} &= t_s + n_\alpha (t_\alpha + t_p) + n_i (t_c + n_{xQ}\,
    t_{xQ})\,,
  \end{align}
\end{subequations}
where $t_s$ is the time for setting up the splitting functions and
mass threshold functions, $n_\alpha$ is the number of different
running couplings that one has, $t_\alpha$ is the time for
initialising the coupling,
%
$n_i$ is the number of PDF initial conditions that one wishes to
consider, $t_i$ is the time to carry out the tabulation and evolution
for a single initial condition, $n_{xQ}$ is the number of points in
$x,Q$ at which one evaluates the full set of flavours once per PDF
initial condition; in the case with cached evolution, $t_p$ is the
time for preparing a cached evolution and $t_c$ is the time for
performing the cached evolution. Finally, $t_{xQ}$ is the time it takes
to evaluate the PDFs at a given value of $(x,Q)$ once the tabulation
has been performed.

Here we focus on $t_s$, $t_i$, and $t_c$. The results can be seen in the
bottom panels of Figures~\ref{fig:sub1}--\ref{fig:sub2} at NNLO and
N$^3$LO respectively.
%
The expected scaling is
$(\text{\texttt{dy}}^2 \text{\texttt{dlnlnQ}})^{-1}$, which for our
choice of $\texttt{dlnlnQ} \propto\texttt{dy}$ reduces to
$1/\text{\texttt{dy}}^3$.
%
That is consistent with what is seen in the plot.
%
Turning to the initialisation time, at NNLO the initialisation time of
$\sim 60-300\,\mathrm{ms}$ dominates over the evolution time across
almost all \texttt{dy} values that we study.
%
It scales slightly more slowly than $1/\texttt{dy}$.

We note that when using cached evolution, the evolution time reaches as
little as $1\,\mathrm{ms}$ for \texttt{dy = 0.2}.
%
Comparing these numbers to those of  
%
\ifreleasenote
Table~2 of Ref.~\cite{Salam:2008qg},
\else
Table~\ref{tab:timings},
\fi
%
which were obtained with 2008 hardware and compilers, we see a
speed-up of roughly a factor 10, which we attribute mainly to
improvements in the hardware.

At N$^3$LO the evolution picture is almost identical to that at NNLO,
but the initialisation times are significantly larger, in the range
$4-15\, \mathrm{s}$, rather than a fraction of a second.
%
This is dominated by the extra time that it takes
to initialise the exact mass threshold functions of
Ref.~\cite{BlumleinCode}.
%
In any application where \hoppet{} either
has to evolve or access the evolved tables multiple times this extra
initialisation is insignificant.
%
Should parametrised expressions of the mass threshold functions become
available, we expect that the initialisation time will be
substantially reduced.

% We conclude that \hoppet{} is competitive compared to other evolution
% codes on the market both in terms of accuracy and speed.
% %
% \gps{We should cite them, and give comparisons where possible}

\section{Python interface}
\label{sec:pyinterface}
From version~2.0.0 \hoppet also includes a Python interface to the
most common evolution routines. It currently includes exactly the same
functionality as the streamlined interface, and can be called in much
the same way. The names of functions and routines are the same as in
the streamlined interface, but stripped of the \ttt{hoppet}
prefix. The interface is installed automatically as \ttt{hoppet} with
\ttt{Cmake} and can be used through \ttt{import hoppet}. For a fully
worked out tabulation example the user should take a look at
\repolink{pyinterface/tabulation_example.py}{pyinterface/tabulation\_example.py}. The
interface uses SWIG (Simplified Wrapper and Interface Generator) which
therefore needs to be available on the system. It can be installed
through most package managers (e.g.\ \texttt{apt install swig},
\texttt{brew install swig}) but
can also be obtained from the SWIG GitHub
\href{https://github.com/swig}{repository}.

One significant difference between the Python interface and the
streamlined interface is that Python does not provide native support
for pointers. For this reason all functions and subroutines that
returned a pointer to a PDF object, through an explicit argument, now
instead return a Python array directly. For instance, where in \CPP{}
one might have the call
%
\begin{lstlisting}
  #include "hoppet.h"
  [...]
  double x = ...;
  double Q = ...;
  double pdf[13];
  [...]
  hoppetEval(x, Q, pdf);
\end{lstlisting}
%
instead in Python one would have
%
\begin{lstlisting}
  import hoppet as hp
  [...]
  x = ...
  Q = ...
  pdf = hp.Eval(x,Q)
\end{lstlisting}
As a consequence the \ttt{Assign}, \ttt{CachedEvolve}, and
\ttt{Evolve} routines, which all expect a PDF interface as an input in
the streamlined interface, now instead take a function of \ttt{x} and
\ttt{Q} that returns a Python array (see \ttt{hera\_lhc} in
\repolink{pyinterface/tabulation_example.py}{tabulation\_example.py}
for an explicit example).

\section{CMake build option}
\label{sec:cmake}
In addition to the standard build option through \ttt{./configure} and
\ttt{make [install]} \hoppet now supports the use of CMake. Although
the old build option will be supported as far as possible, the CMake
build option is the main build option starting from version
2.0.0 of \hoppet.

For a typical user it will be enough to invoke the following lines
from the main directory
\begin{lstlisting}
  mkdir build; cd build
  cmake ..
  make [-j]
  make install [-j]
\end{lstlisting}
This will compile and install \hoppet along with both the streamlined
and Python interfaces. Note that \ttt{make install} will typically
install in a location that requires root privileges, unless a user has
specified a custom prefix (through
\ttt{-DCMAKE\_INSTALL\_PREFIX=/install/path}). A number of options can
be passed to CMake. They are documented in
\repolink{CMakeLists.txt}{CMakeLists.txt} and can be printed on screen
by a call to \ttt{cmake -LH ..} from the \ttt{build} directory.

Of particular note to most users are
\begin{lstlisting}
  option(HOPPET_USE_EXACT_COEF    "Use exact coefficient functions"  OFF)
  option(HOPPET_BUILD_EXAMPLES    "Build examples" ON)
  option(HOPPET_ENABLE_TESTING    "Enable testing. Requires building the examples." ON)
  option(HOPPET_BUILD_BENCHMARK   "Build benchmark." ON)
  option(HOPPET_BUILD_PYINTERFACE "Build python interface." ON)
  option(HOPPET_ENABLE_FPES       "Enable trapping for the floating point exceptions." OFF)
  option(HOPPET_BUILD_TESTDIR     "Enable build of the code in the test directory." OFF)
\end{lstlisting}
The can be set when call \ttt{cmake} in the usual way,
e.g.\ \ttt{cmake ..\ -DHOPPET\_USE\_EXACT\_COEF=ON} to compile \hoppet
with the exact coefficient functions.

\section{Saving LHAPDF grids}
\label{sec:lhapdf}
A minor new feature of this release is the possibility to save a
\hoppet{} table in the form of an LHAPDF6~\cite{LHAPDF} grid.
%Access to this functionality is also provided in the streamlined interface.
% GPS -> removed this because it's said below
The main routine has the following structure
\begin{lstlisting}
  subroutine WriteLHAPDFFromPdfTable(table, coupling, basename, pdf_index, &
                                 & iy_increment, flav_indices, flav_pdg_ids, flav_rescale)                                
    use qed_objects
    type(pdf_table),        intent(in) :: table
    type(running_coupling), intent(in) :: coupling
    character(len=*),       intent(in) :: basename
    integer,                intent(in) :: pdf_index
    integer,  optional,     intent(in) :: iy_increment
    integer,  optional,     intent(in) :: flav_indices(:), flav_pdg_ids(:)
    real(dp), optional,     intent(in) :: flav_rescale(:)
\end{lstlisting}
A user has to provide a \ttt{table} and associated \ttt{coupling}
object along with a string \ttt{basename} and the \ttt{pdf\_index} as
needed by LHAPDF.
%
If \ttt{pdf\_index} is equal to 0 then the routine
outputs the content of the table in \ttt{basename\_0000.dat} and
writes a template \ttt{basename.info} again in the LHAPDF
format.
%
\hoppet{} fills most of the entries in the \ttt{.info}
file, but a few need to be edited manually by the user.
%
For any other value of \ttt{pdf\_index} only the corresponding
\ttt{.dat} file gets written.

By default the code uses the same grid spacing as in the internal
\hoppet{} table (\ttt{iy\_increment = 1}) and prints all the possible
flavours of the PDF, even if they are zero.
%
The user can overwrite
this default behaviour by providing an array of flavours and their pdg
values.
%
If \ttt{iy\_increment > 1} a coarser grid is provided by
skipping over \ttt{iy\_increment - 1} points in the grid. The
\ttt{flav\_rescale} is currently needed for the lepton PDF which is
provided as the sum over flavour and anti-flavour and therefore needs
an extra factor half.
%
The \ttt{flav\_rescale} array only needs to be
provided if the user is also providing the array of flavours and pdf
values.

Finally the routine can also be accessed from the streamlined
interface for \CPP{} or Python usage. In this case the routine is
significantly simplified and only takes \ttt{basename} and the
\ttt{pdf\_index} arguments, for instance in \CPP{} like this
\begin{lstlisting}
  #include "hoppet.h"
  #include <string>
  ...
  int main () {
    ...
    const std::string basename = ...;
    const int pdf_index = ...;
    hoppetWriteLHAPDFGrid(basename, pdf_index);
    ...
  }
\end{lstlisting}
The routine writes the contents of the streamlined interface
\ttt{tables(0)} and hence requires that this object has been filled
either through a call to \ttt{hoppetAssign} or \ttt{hoppetEvolve}.

%%% Local Variables:
%%% TeX-master: "HOPPET-v2.0-release.tex"
%%% End:
